% !TeX spellcheck = de_DE
\documentclass{beamer}

\usetheme{Madrid}
\usecolortheme{beaver}

% Navigation Symbols off
\beamertemplatenavigationsymbolsempty{}
%\usenavigationsymbolstemplate{}
\setbeamercolor*{item}{fg=red}
\setbeamercolor{itemize item}{fg=red} % all frames will have red bullets
% Watermark
\usebackgroundtemplate{%
	\rule{0pt}{\paperheight}%
	\hspace*{\paperwidth}%
	\makebox[0pt][r]{\includegraphics[width=60mm]{\resWatermark}}}

\usepackage[utf8]{inputenc}
\usepackage[german]{babel}
\usepackage{latexsym}
\usepackage{stmaryrd}
\usepackage{amsmath}
\usepackage{amssymb}
\usepackage{amsxtra}
\usepackage{amsthm}
\usepackage{hyperref}
\usepackage{graphicx}
\usepackage{tikz}
\usepackage[square,sort,comma,numbers]{natbib}

% Logos
\newcommand{\resLogo}{../images/logo}
\newcommand{\resFBI}{../images/logo_fbi}
\newcommand{\resLogoSingle}{../images/logo_uhh}
\newcommand{\resWatermark}{../images/watermark}

% Custom commands
\newcommand{\gqq}[1]{\glqq#1\grqq} % encloses words in German quotes

\title{Ansätze und Verfahren der Datenmigration}
\subtitle{Datenmigration im Kontext des Reengineering}
\author[Schenkemeyer, Fechner]{\includegraphics[height=1cm]{\resLogo}\\Julian Schenkemeyer, Tobias Fechner\\ \texttt{\footnotesize\{5schenke,1fechner\}@informatik.uni-hamburg.de}}
\institute[UHH]{Universität Hamburg}
\date{\today}

% === DOCUMENT ===
\begin{document}
	
	% === TITLE ===
	\begin{frame}
		\maketitle
	\end{frame}
	
	\addtobeamertemplate{frametitle}{}{%
		\begin{tikzpicture}[remember picture, overlay]
		\node[anchor=north east,yshift=2pt, xshift=2pt] at (current page.north east) {%
			\includegraphics[height=.75cm]{\resFBI}\hspace{3pt}%
			\includegraphics[height=.75cm]{\resLogoSingle}%
		};
		\end{tikzpicture}
	}
	
	% === AGENDA ===
	
	\begin{frame}
		\frametitle{Agenda}
		\tableofcontents
	\end{frame}
	
	% ===========================================================================================================
	% ======= BEGIN CONTENT =====================================================================================
	% ===========================================================================================================
	
	\section{Datenmigration im Kontext des Reengineering}
	
	\begin{frame}
		\frametitle{Datenmigration}
		
		\textit{''Data migration is the selection, perparation, extraction, transformation and
		permanent movement of appropriate data that is of the right quality to the
		right place at the right time and the decommissioning of legacy data stores''} \cite{morris-2012}
	\end{frame}
	
	\begin{frame}
		\frametitle{H"aufige Riskien der Datenmigration}
		
		\begin{itemize}
			\item Technologiezentrierung
			\item Mangel an Spezialisten
			\item Unterschätzen der Datenmigration
			\item Problemzuweisung
		\end{itemize}
	\end{frame}
	
	% ===========================================================================================================
	\section{Generelles Vorgehen}
	
	\begin{frame}
		\frametitle{"Uberblick}
		
		Prozessmodell einer Datenmigration
		
		\begin{enumerate}
			\item Initialisierung
			
			\item Migrationsentwicklung
			
			\item Testen
			
			\item Umstellung auf das neue System
			
		\end{enumerate}
	\end{frame}
	
	\begin{frame}
		\frametitle{Abschnitt 1 - Initialisierung}
		
		\begin{enumerate}
			\item \textbf{Initialisierung}
			\begin{itemize}
				\item Um was für ein Legacy-System handelt es sich?
				\item Welche Daten sollen auf das neue System übertragen werden?
				\newline
				\item Entscheidung dar"uber welche Vorgehensweise verwendet werden soll
				\item initiale Aufwandseinschätzung
				\item Einrichten einer 
			\end{itemize}
			
			\item Migrationsentwicklung
			
			\item Testen
			
			\item Umstellung auf das neue System
			
		\end{enumerate}
	\end{frame}
	
	\begin{frame}
		\frametitle{Abschnitt 2 - Initialisierung}
		
		\begin{enumerate}
			\item Initialisierung
			
			\item \textbf{Migrationsentwicklung}
			\begin{itemize}
				\item "Uberspielen eines Datenbackup auf die Migrationsplattform
				\item Vertiefung der Analyse des Datenbestands des Legacy-Systems
				\item Entwicklung der Migrationsskripte und -werkzeuge
				\item evtl. Durchf"uhrung von Data-Cleansing
			\end{itemize}
			
			\item Testen
			
			\item Umstellung auf das neue System
			
		\end{enumerate}
	\end{frame}
	
	\begin{frame}
		\frametitle{Abschnitt 3 - Testen}
		
		\begin{enumerate}
			\item Initialisierung
			
			\item Migrationsentwicklung
			
			\item \textbf{Testen}
			\begin{itemize}
				\item "Uberpr"ufung der Funktionalit"at der Migrationsskripte
			\end{itemize}
			
			\item Umstellung auf das neue System
			
		\end{enumerate}
	\end{frame}
	
	\begin{frame}
		\frametitle{Abschnitt 4 - Umstellung auf das neue System}
		
		\begin{enumerate}
			\item Initialisierung
			
			\item Migrationsentwicklung
			
			\item Testen
						
			\item \textbf{Umstellung auf das neue System}
			\begin{itemize}
				\item Je nach gew"ahlter Vorgehensweise verl"auft dieser Abschnitt unterschiedlich
				\newline
				\item Erstellung eines Erfahrungsberichts
			\end{itemize}
			
		\end{enumerate}
	\end{frame}
	% ===========================================================================================================
	\section{Strategien zur Datenmigration}
	
	\begin{frame}
		\frametitle{Stratgien}
		
		\begin{itemize}
			\item Strategien auf unterschiedlichen Ebenen
			\item "Anderungen in Daten ziehen "Anderungen in nutzenden Anwendungen nach sich
			\item Ziel ist die Erhaltung von Daten
			\item Unterschiedliche Schwerpunkte erfordern unterschiedliches technisches und fachliches Wissen
		\end{itemize}
	\end{frame}
	
	\subsection{Datenbankebene}
	
	\begin{frame}
		\frametitle{Datenbankebene}
		
		\begin{itemize}
			\item \textbf{Physisch}
				\begin{itemize}
					\item "Ubertrag von Daten
					\item H"aufig Portierung auf neues DBMS, keine fachliche Analyse
					\item Niedriger Aufwand, Automatisierbar
				\end{itemize}
			\item \textbf{Konzeptuell}
				\begin{itemize}
					\item Neukonzeptionierung der Datenhaltung
					\item Fachliche Betrachtung
					\item Hoher Aufwand
				\end{itemize}
		\end{itemize}
	\end{frame}	
	
	\begin{frame}
		\frametitle{Konzeptuelle Konvertierung}
		
		\centering
		\includegraphics[height = 6cm]{../images/strategies_fig_02b.png}\\
		\tiny Quelle: \cite{henrard-2002}
	\end{frame}
	
	\subsection{Anwendungsebene}
	
	\begin{frame}
		\frametitle{Anwendungsebene}
		
		\begin{itemize}
			\item \textbf{Einsatz eines Wrappers} 
				\begin{itemize}
					\item Wrapper verwaltet Zugriff auf neue Datenquelle
					\item Anwendung benutzt Wrapper statt neue Datenquelle
					\item Nutzung durch Wrapper f"ur Anwendung nur minimal ver"andert
					\item Niedriger Aufwand, erfordert wenig Kenntnisse der Anwendung
				\end{itemize}
			\item \textbf{Anpassung von Statements} 
				\begin{itemize}
					\item Aufrufe der Datenquelle werden angepasst
					\item Mittlerer Aufwand, erfordert mittlere Kenntnisse der Anwendung
				\end{itemize}
			\item \textbf{Anpassung der Zugriffslogik}
				\begin{itemize}
					\item Zugriffslogik wird angepasst
					\item Technische M"oglichkeiten der neuen Quellen k"onnen genutzt werden
					\item Hoher Aufwand, erfordert umfangreiche Kenntnisse der Anwendung
				\end{itemize}
		\end{itemize}
	\end{frame}
	
	\begin{frame}
		\frametitle{Vorgehen zur Einf"uhrung eines Wrappers}
		
		\centering
		\includegraphics[height = 7cm]{../images/large_scale_fig_01.png} \\
		\tiny Quelle: \cite{henrard-2008}
	\end{frame}
	
	% ===========================================================================================================
	\section{Einf"uhrungsstrategien der Datenmigration}
	
	\subsection{Big-Bang-Ansatz}
	
	%TODO
	
	\subsection{Chicken-Little-Ansatz}
	
	%TODO
	
	\subsection{Butterfly-Ansatz}
	
	%TODO
	
	% ===========================================================================================================
	\section{Fazit}
	
	\begin{frame}
		\frametitle{Fazit}
		
		\begin{itemize}
			\item TODO %TODO
		\end{itemize}
	\end{frame}
	
	% ===========================================================================================================
	% ======= END CONTENT =======================================================================================
	% ===========================================================================================================
	
	% === QUELLEN ===
	%    \section{Quellen}
	
	\begin{frame}
		\frametitle{Quellen}
						
		\bibliographystyle{plaindin}
		\bibliography{../bib/mproj}
	\end{frame}
	
\end{document}
