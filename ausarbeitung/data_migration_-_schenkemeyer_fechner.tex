% !TeX spellcheck = de_DE
\documentclass[11pt]{scrartcl}
\usepackage[utf8]{inputenc}
\usepackage[german]{babel}
\usepackage[T1]{fontenc}
\usepackage{latexsym}
\usepackage{stmaryrd}
\usepackage{amsmath}
\usepackage{amssymb}
\usepackage{amsxtra}
\usepackage[square,sort,comma,numbers]{natbib}
\usepackage{listings}
\usepackage{url}
\usepackage{hyperref}
\usepackage{graphicx}
\usepackage{color}

\newcommand{\ncite}{\footnote{Citation needed}}
\newcommand{\lb}{\\ \\}

\newtheorem{theorem}{Theorem} 
\newtheorem{definition}[theorem]{Definition} 

\newif\iffull
\fulltrue

\definecolor{mygray}{rgb}{0.95,0.95,0.95}
\lstset{ %
	backgroundcolor=\color{mygray}}

\setlength{\parindent}{0em}

\begin{document}
	
\iffull
% === TITLE PAGE ===
\title{Data Migration} 

\subtitle{Untertitel}

%\author{Julian Schenkemeyer, Tobias Fechner\\
%	{\texttt{\{5Schenke,1fechner\}@informatik.uni-hamburg.de}}}

\date{Modul Software Reengineering 2015/2016\\
  \small Fachbereich Informatik\\ 
  Arbeitsbereich Softwarekonstruktion \& Werkzeuge\\ 
  Universit"at Hamburg\\[4mm]
  \today}

\maketitle

% === ABSTRACT ===
\begin{abstract}
	\small\noindent\textbf{Abstract}

	\noindent Abstract 
\end{abstract}

\newpage
\tableofcontents
\newpage

\fi

% ===========================================================================================================
% ======= BEGIN CONTENT =====================================================================================
% ===========================================================================================================

%TODO Include Sections!

\nocite{morris-2012}
\nocite{henrard-2002}
\nocite{behm-1997}
\nocite{datamigrations}
\nocite{ackermann-2005}
\nocite{wagner-2011}
\nocite{wu-1997}
\nocite{rahm-2010}
\nocite{alhajj-2001}

\section{Einleitung}

\begin{itemize}
	\item Was ist Datenmigration
	\item Es gibt verschiedene Ans"atze und Einf"uhrungsstrategien
	\item Warum bei Reengineering
\end{itemize}

\section{Motivation}
%http://www.dtic.upf.edu/~jbisbal/publications/icsc97.pdf
%http://www.aifb.kit.edu/images/2/27/2002_14_Stojanovic_A_reverse_engin_1.pdf 	%Kann eventuell verwendet werden, als inspiration f"ur Szenario?
%http://edepositireland.ie/bitstream/handle/2262/27040/The%20Butterfly%20Methodology%20a%20gateway-free%20approach%20for%20migrating%20legacy%20information%20systems.pdf?sequence=1&isAllowed=
%http://www.dtic.upf.edu/~jbisbal/publications/datasem97.pdf
%http://csis.pace.edu/~marchese/CS775/Proj1/legacyinfosys_directions.pdf
%http://www.tara.tcd.ie/bitstream/handle/2262/27050/An?sequence=1
%http://www.oracle.com/technetwork/middleware/oedq/successful-data-migration-wp-1555708.pdf
%http://searchsmbstorage.techtarget.com/tip/Data-migration-strategies-and-best-practices
%https://pure.fundp.ac.be/ws/files/168599/wcre02.pdf
%http://www.dtic.upf.edu/~jbisbal/publications/datasem97.pdf
%http://csis.pace.edu/~marchese/CS775/Proj1/legacyinfosys_directions.pdf
%http://www.tara.tcd.ie/bitstream/handle/2262/27050/An?sequence=1
%http://ieeexplore.ieee.org/stamp/stamp.jsp?tp=&arnumber=271615 ==> IEEE Xplore Data Migration von Cheong Youn, Cyril S. Ku
%http://www.oracle.com/technetwork/middleware/oedq/successful-data-migration-wp-1555708.pdf

\begin{itemize}
	\item Warum ist Datenmigration besonders wichtig bei Reengineering?
	\item General Scenario
	\item Ans"atze grob einf"uhren
	\item Migration als Erhaltung von Daten
	\item Etwa wichtige Kunden- oder Transaktionsdaten von Banken
	\item Daten werden weiterhin verwendet, System m"ussen (technisch oder fachlich) erneuert werden
	\item Die bereits vorhandenen Daten werden u.U. anders genutzt oder erfordern Anpassungen
	\item Risiken der Datenmigration
\end{itemize}

\section{Kategorien Datenmigration}
%https://pure.fundp.ac.be/ws/files/168599/wcre02.pdf

Jeweils:
\begin{itemize}
	\item Wessen Aufgabe ist die Migration?
	\item Auswirkung auf umliegende Systeme (Folgen)?
	\item Vorteile?
	\item Nachteile?
\end{itemize}

\subsection{Storage Migration}

\begin{itemize}
	\item Neue physische Hardware
	\item Jemand kauft einen neuen Server -> unter Umst"anden neues Schreiben der Daten auf physischen Datentr"ager
\end{itemize}

\subsection{Database Migration}

\begin{itemize}
	\item Neue Version von DBMS etc.
	\item Etwa Upgrade Oracle 7.0 -> 11.0: Unterschiedliche Verwaltung von Speicher und Administration
\end{itemize}

\subsection{Application Migration}

\begin{itemize}
	\item Neues DB-System etc. (Etwa Oracle -> MSSql)
	\item Andere Formate, Statements, Zugriffe etc.
\end{itemize}

\subsection{Business Process Migration}

\begin{itemize}
	\item Anpassung von Daten an Business-Prozesse
	\item Etwa Spaltung eines Unternehmens: Vorher alle Daten in einer DB, nach Spaltung geh"oren die Daten rechtlich entweder dem einen oder anderen Unternehmen und m"ussen entsprechend getrennt werden.
	\item Datenhaltung bildet von Business-Prozesse ab und muss entsprechend migriert werden
\end{itemize}

\section{Planung und Durchführung}
%http://www.information-management.com/specialreports/20040518/1003611-1.html
%http://searchsmbstorage.techtarget.com/tip/Data-migration-strategies-and-best-practices
%https://pure.fundp.ac.be/ws/files/168599/wcre02.pdf
%http://www.dtic.upf.edu/~jbisbal/publications/datasem97.pdf
%http://dc-pubs.dbs.uni-leipzig.de/files/Rahm2000DataCleaningProblemsand.pdf
%http://csis.pace.edu/~marchese/CS775/Proj1/legacyinfosys_directions.pdf
%http://www.oracle.com/technetwork/middleware/oedq/successful-data-migration-wp-1555708.pdf

\begin{itemize}
	\item Vorstellung der einzelnen Phasen der Datenmigration
	\begin{itemize}
		\item Data Assessment
		\begin{itemize}
			\item Welche Datenquellen existieren
			\item Welche Daten müssen migriert werden
			\item Wie wird migriert(Methode) und wie wird anschließend die Validation durchgeführt
		\end{itemize}
		
		\item Data Cleansing
		\begin{itemize}
			\item Was wird nicht ben"otigt
			\item Aufr"aumen der bestehenden Daten
			\item "Uberpr"ufen der Datenqualität
		\end{itemize}
		
		\item Test Extract \& Load
		\begin{itemize}
			\item Erstellen der notwendigen Skripte zur automatischen Durchf"uhrung der Datenmigeration 
			\item Testlauf mit Teildatensatz
		\end{itemize}
		
		\item Final Extract \& Load
		\begin{itemize}
			\item Durchf"uhrung der Datenmigration mit jeweiliger Vorgehensweise
		\end{itemize}
		
		\item Migration Validation
		\begin{itemize}
			\item Abschließende "Uberpr"ufung der migrierten Daten
			\item eventuelle Korrekturen
		\end{itemize}
		
		\item Post Migration Activities
		\begin{itemize}
			\item Anpassung Dokumentation und Referenzen
			\item Lernen fürs nächste Mal
		\end{itemize}
		
	\end{itemize}
\end{itemize}

\section{Vorgehensweisen zur Datenmigration}
%http://www.information-management.com/specialreports/20040525/1003961-1.html
%http://edepositireland.ie/bitstream/handle/2262/27040/The%20Butterfly%20Methodology%20a%20gateway-free%20approach%20for%20migrating%20legacy%20information%20systems.pdf?sequence=1&isAllowed=y
%http://www.dtic.upf.edu/~jbisbal/publications/icsc97.pdf
%https://pure.fundp.ac.be/ws/files/168599/wcre02.pdf
%http://www.dtic.upf.edu/~jbisbal/publications/datasem97.pdf
%http://csis.pace.edu/~marchese/CS775/Proj1/legacyinfosys_directions.pdf

\subsection{Big-Bang-Approach}

\begin{itemize}
	\item Migration in einem Schritt
	\item alte Datenbanken, Schemata, etc. wegwerfen
	\item Komplette Modell werden ausgetauscht
\end{itemize}

\subsection{Phased-Approach}

\begin{itemize}
	\item Alte und neue Daten pflegen
	\item Hohe Redundanzen
	\item Alten Daten bleiben zun"achst erhalten
	\item Oder auch: Nur Teile der Datenmodell/ Datenbanken werden migriert
\end{itemize}

\subsection{Butterfly-Approach}

\begin{itemize}
	\item Kein Gateway f"ur Daten"ubertragung
\end{itemize}

\section{Fazit}

\iffull
% === BIB ===
\newpage
\bibliographystyle{alphadin}
\addcontentsline{toc}{section}{\bibname}

\bibliography{../bib/mproj}
\fi
\end{document}