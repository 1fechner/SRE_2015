% !TeX spellcheck = de_DE
\section{Vorgehensweisen zur Datenmigration}
\label{chapter:vorgehensweisen}
%TODO Julian
%http://www.information-management.com/specialreports/20040525/1003961-1.html
%http://edepositireland.ie/bitstream/handle/2262/27040/The%20Butterfly%20Methodology%20a%20gateway-free%20approach%20for%20migrating%20legacy%20information%20systems.pdf?sequence=1&isAllowed=y
%http://www.dtic.upf.edu/~jbisbal/publications/icsc97.pdf
%https://pure.fundp.ac.be/ws/files/168599/wcre02.pdf
%http://www.dtic.upf.edu/~jbisbal/publications/datasem97.pdf
%http://csis.pace.edu/~marchese/CS775/Proj1/legacyinfosys_directions.pdf

%Quellen hinzufügen

Das vom Unternehmen angewandte Prozessmodell beinhaltet auch die Wahl einer Vorgehensweise, die den technischen Ablauf der Datenmigration beschreibt. Da die gewählte Vorgehensweise den weiteren Ablauf im Prozessmodell beeinflusst, wird diese in einer frühen Phase nachdem die Kerndaten von Legacy- und dem neuen System analysiert wurden ausgewählt. Eine spätere Änderung dieser Entscheidung ist nicht oder nur sehr schwer möglich. Auch hier gilt das nicht Vorgehensweise sich für alle Unternehmen zur Datenmigration eignen. In diesen Kapitel werden aus diesem Grund drei dieser Vorgehensweisen mit ihren jeweiligen Vor- und Nachteil vorgestellt. 

\subsection{Big-Bang-Ansatz}

Der Big-Bang oder auch als Cold Turkey bekannter Ansatz ist vermutlich der älteste Ansatz zur Datenmigration. Bei ihm handelt es sich um eine komplette Neuentwicklung des Legacy-Systems. Das neuentwickelte System bietet alle Funktionen des Legacy-Systems, basiert jedoch auf neuer Hardware sowie moderner Entwicklungsmethoden und Datenbanken.\citep[S. 105]{bisbal-1999}
\lb
Die Anforderungen, die das neuentwickelte System erfüllen muss, zunächst aus den verschiedenen Quellen zusammengetragen werden. Dazu gehören unteranderem die Dokumentation sowie der Programmcode des Legacy-Systems.\citep[S. 2]{brodie-1993} Zu Problemen kann es hier allerdings kommen wenn die Dokumentation veraltet ist und nicht mehr alle Funktionalitäten des Legacy-Systems beschreibt. Undokumentierte Funktionen und deren Abhängigkeiten müssen in diesem Fall dann direkt aus dem Anwendungen und dem Programmcode des Legacy-Systems abgeleitet werden.\citep[S. 2]{brodie-1993} 
\lb
Nach Abschluss der Entwicklung des neuen System und ausführlichem testen sämtlicher Funktionalitäten kann mit der Datenmigration begonnen werden. Beim Big-Bang-Ansatz sieht diese Datenmigration eine Pausierung sämtlicher vom Legacy-System gestützter Geschäftprozesse vor.\citep[S. 4]{wuLawless-1997} Während dieser Pausierung wird ein Dump des gesamten Datenbestands des Legacy-Systems angelegt. Dieser wird anschließend mittels der hierfür entwickelte Migrationsskripte in das neue System übertragen.\citep[S. 3]{brodie-1993} Bis die Datenmigration abgeschlossen ist, kann weder das Neue noch das Legacy-System im Unternehmen eingesetzt werden. Daraus resultiert ein Zeitraum, in dem das Unternehmen ohne das ein geschäftskritisches System auskommen muss und somit stark in seiner Geschäftsfähigkeit eingeschränkt ist.\citep[S. 3f]{brodie-1993}
\lb
Die Vorteile diese Ansatzes beschränken sich auf das verbesserte Programmverständnis, die Performance und die bessere Wartbarkeit des neu entwickelten Systems, nachteilig hingegen ist ein hohes Versagensrisiko.\citep[S. 105]{bisbal-1999} Das hohe Versagensrisiko des Big-Bang-Ansatz setzt sich aus mehreren Faktoren zusammen. Einer dieser Faktoren ist die Identifizierung der Anfordungen an das neue System aus der Dokumentation und dem Programmcode des Legacy-System. Je länger das Legacy-System in Betrieb war desto wahrscheinlicher ist auch dessen Dokumention veraltet. Einzelne Funktionalitäten und Abhängigkeiten innerhalb des Legacy-Systems könnten falsch oder nur unzureichend dokumentiert sein.\citep[S. 2]{brodie-1993} Darüber hinaus bestehen auch die üblichen Probleme eines Softwareprojekts. Anforderungen können sich im laufe der Entwicklung ändern, Fristen können überzogen werden und die Kosten für die Entwicklung des neuen Systems in die höhe treiben.\citep[S. 2f]{brodie-1993} Neben dem hohen Versagensrisiko besteht beim Big-Bang-Ansatz allerdings auch die Problematik mit der Datenmigration, die für eine Unterbrechung aller vom Legacy- bzw. dem neuen System unterstützten Geschäftsprozesse sorgt.\citep[S. 4]{wuLawless-1997}
\lb
Zusammenfassend bietet der Big-Bang-Ansatz auf der einen Seite eine unkomplizierte Datenmigration. Auf der anderen Seite jedoch stehen ein hohes Versagensrisiko sowie eine durch die eine Ausfallzeit während des Übergängs zum neuen System in dem im Unternehmen gar nicht und nur sehr eingeschränkt weitergearbeitet werden kann. Der Big-Bang-Ansatz kann somit nicht in jeden Unternehmen, sondern nur in einige wenigen eingesetzt werden. 

\subsection{Chicken-Little-Ansatz}

% Hier noch Grafik einbauen

Der Chicken-Little-Ansatz stellt eine weitere Vorgehensweise zur Datenmigration dar. Entwickelt wurde dieser Ansatz von Michaell Brodie und Michael Stonebraker im Rahmen des 1991 begonnenen DARWIN-Projekts der University of California in Berkeley.\citep{zoulafy-2002} Der Chicken-Little-Ansatz stellt im Gegensatz zum Big-Bang-Ansatz ein inkrementelles Vorgehen zur Datenmigration dar. Durch das inkrementelle Vorgehen sollte Schwächen des Big-Bang-Ansatz beseitigt werden. So können bei Chicken-Little beispielsweise Ausfallzeiten der von Legacy- bzw. neuen System unterstüzten Geschäftsprozesse während der Datenmigration vermieden oder zumindest minimal gehalten werden.\citep{zoulafy-2002}
\lb
Das inkrementelle Vorgehen ermöglicht hierbei, dass das Legacy-System nicht komplett aufeinmal sondern in kleineren Modulen neu entwickelt werden kann. So beginnt das neue System zunächst mit nur wenigen Funktionalitäten des Legacy-System. Mit der Zeit werden dann immer weiter Funktionalitäten des Legacy-System auf das neue System übertragen.\citep[S. 2]{wuLawless-1997} 
\lb
Um auch während der Datenmigration die Weiterarbeit zu ermöglichen kommt beim Chicken-Little-Ansatz ein sogenanntes Gateway zum Einsatz. Dieses Gateway verbindet Legacy- und das neue System miteinander und managt deren Kommunikation untereinander. Somit können das Neue und das Legacy-System solange parallel ausgeführt werden bis das neue System alle Daten enthalt und alle Funktionalitäten übernehmen kann\citep[S. 2]{wuLawless-1997} Die neuen Daten, die während der Zeit entstehen in der das neue System noch nicht alle Funktionen übernommen hat, werden dabei auf dem für die Funktion zuständigen System gespeichert. Folglich werden Daten, die durch bereits ans neue System übertragene Funktionen erstellt worden sind auch dort gespeichert und nicht mehr auf dem Legacy-System.\citep[S. 2]{wuLawless-1997} 
\lb
Das Gateway erfüllt nun zwei Aufgaben dem Übertragen der Daten an das neue System. Zum einen ist dies das Bereitstellen der noch nicht migrierten Daten an das neue System und zum Anderen das Bereitstellen der bereits migrierten Daten an das Legacy-System. Ersteres wird auch als rückwärts Gateway(engl.: reverse Gateway) und letzteres als vorwärts Gateway(engl.: forward Gateway) bezeichnet.\citep[S. 2]{wuLawless-1997} Je nachdem wie stark die Datenbankschemata vom Legacy- und neuen System sich unterscheiden stellen vorwärts und rückwärts Gateway unterschiedlich komplexe Module dar.  Mit steigender Komplexität dieser Module sinkt allerdings auch die Performanz des System während dieser Migrationphase.\citep[S. 109]{bisbal-1999}
\lb
Genau wie der Big-Bang-Ansatz bietet der Chicken-Little-Ansatz die erwarteten Verbesserungen in Performance, Wartbarkeit sowie im Verständnis des Systems.\citep[S. 108]{bisbal-1999} Allerdings fällt beim Chicken-Little-Ansatz keine Ausfallzeiten, in denen weder mit Legacy- noch mit dem neuen System gearbeitet werden kann an. Ermöglicht wird dies durch den Einsatz des Gateway, wodurch auch während der Migrationsphase mit dem gesamten System gearbeitet werden kann.\citep[S. 2]{wuLawless-1997} Darüber hinaus hat der Chicken-Little-Ansatz noch weitere Vorteile. So muss beispielsweise nicht das gesamte Legacy-System auf einmal ersetzt werden, stattdessen kann dieses mit diesem Ansatz Funktion für Funktion erneuert werden. Wodurch die Entwicklung des neuen Systems erleichtert werden kann.\citep[S. 3]{brodie-1993} Die inkrementelle Entwicklung des neuen System bietet aber auch Vorteile in der Fehlerbehandlung. So müssen bei Auftreten von Fehlern nur die entsprechenden Schritte wiederholt werden. Wenn zum Beispiel ein Problem bei der Datenmigration vom entwickelten Modul auftritt, muss nach der Fehlerbehebung nur die Datenmigration von diesem Modul wiederholt werden. Durch die Fehlerbehebung gewonnenen Kenntnisse können ebenfalls in den nächsten Schritten genutzt werden um bekannte Probleme zu vermeiden bevor diese entstehen.\citep[S. 3]{brodie-1993} 
\lb
% je nach LIS unterschiedlich schwer durchzuführen
Neben den genannten Vorteilen hat der Chicken-Little-Ansatz aber auch ein paar Nachteile aufzuweisen. Zwar erlaubt der Einsatz des Gateway während der Migrationsphase mit einem kombinierten System zu arbeiten, dafür stellt die Entwicklung des Gateway eine Herausforderung dar. Daten, die auf beiden Systemen vorhanden sind müssen konsistent gehalten werden und um die Interoperabilität der beiden Systeme zu ermöglichen, muss das Gateway wissen wie die Daten in den unterschiedlichen Schemata gespeichert sind. Je unterschiedlicher die Datenbankschemata von Legacy- und neuem System sind, desto komplexer wird diese Aufgabe.\citep[S. 2f]{wuLawless-1997} Darüber hinaus muss bei komplexeren Gateways auch mit mehr Performanzeinbußen gerechnet werden, wenn das vor- bzw. rückwärts Gateway zum Einsatz kommen.\citep[S. 109]{bisbal-1999}
\lb
Alles in allem betrachtet bietet der Chicken-Little-Ansatz viele Vorteile. Die inkrementelle Herangehensweise minimiert viele Risiken und Probleme, wie z.B. die Ausfallzeiten, des Big-Bang-Ansatzes, wodurch sich Chicken-Little auch zum Einsatz größeren Unternehmen eignet. Dennoch muss bedacht werden, dass Chicken-Little auch schnell sehr komplex werden kann.

\subsection{Butterfly-Approach}

%\begin{itemize}
%	\item Kein Gateway f"ur Daten"ubertragung
%\end{itemize}

