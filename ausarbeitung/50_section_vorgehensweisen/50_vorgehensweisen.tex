% !TeX spellcheck = de_DE
\section{Vorgehensweisen zur Datenmigration}
\label{chapter:vorgehensweisen}
%TODO Julian
%http://www.information-management.com/specialreports/20040525/1003961-1.html
%http://edepositireland.ie/bitstream/handle/2262/27040/The%20Butterfly%20Methodology%20a%20gateway-free%20approach%20for%20migrating%20legacy%20information%20systems.pdf?sequence=1&isAllowed=y
%http://www.dtic.upf.edu/~jbisbal/publications/icsc97.pdf
%https://pure.fundp.ac.be/ws/files/168599/wcre02.pdf
%http://www.dtic.upf.edu/~jbisbal/publications/datasem97.pdf
%http://csis.pace.edu/~marchese/CS775/Proj1/legacyinfosys_directions.pdf

%Quellen hinzufügen
% neue System -> Zielsystem + Definition?

Das vom Unternehmen angewandte Prozessmodell beinhaltet ebenso die Wahl einer Vorgehensweise, welche den technischen Ablauf und die Einf"uhrung der Datenmigration beschreibt. Da die gewählte Vorgehensweise den weiteren Verlauf des Prozessmodells beeinflusst, wird diese in einer frühen Phase, nachdem die Kerndaten von Legacy- und neuem System analysiert wurden, ausgewählt. Eine spätere Revision dieser Entscheidung ist nicht oder nur sehr schwer möglich. Auch hier gilt, dass sich nicht jede Vorgehensweise in jedem Kontext f"ur die Durchf"uhrung einer Datenmigration eignet. Im Folgenden werden zu diesem Zweck drei Vorgehensweisen mit ihren jeweiligen Vor- und Nachteilen vorgestellt. 

\subsection{Big-Bang-Ansatz}

%TODO Bild??

Der \textit{Big-Bang} oder auch als \textit{Cold Turkey} bekannte Ansatz ist wohl das älteste Model zur Durchf"uhrung einer Datenmigration. Bei diesem handelt es sich um eine komplette Neuentwicklung des Legacy-Systems. Das neu entwickelte System bietet alle Funktionen des Legacy-Systems, basiert jedoch auf neuer Hardware sowie moderner Entwicklungsmethoden und Datenbanken \citep[S.~105]{bisbal-1999}.
\lb
Die Anforderungen, welche das neu entwickelte System erfüllen muss, sind zunächst aus den verschiedenen Quellen zusammenzutragen. Zu diesen gehören unter anderem die Dokumentation sowie der Programmcode des Legacy-Systems \citep[S.~2]{brodie-1993}. Dies kann zu Problemen f"uhren, sollte die Dokumentation veraltet sein und nicht mehr alle Funktionalitäten des Legacy-Systems beschreiben. Undokumentierte Funktionen und deren Abhängigkeiten müssen in diesem Fall direkt aus dem Anwendungssystem und dem Programmcode des Legacy-Systems abgeleitet werden \citep[S.~2]{brodie-1993} .
\lb
Nach Abschluss der Entwicklung des neuen Systems und ausführlichem Testen sämtlicher Funktionalitäten kann mit der Datenmigration begonnen werden. Der Big-Bang-Ansatz sieht diese Datenmigration als eine zeitweise Unterbrechung sämtlicher vom Legacy-System gestützter Geschäftsprozesse vor \citep[S.~4]{wuLawless-1997}. Während dieser Unterbrechung wird ein Dump des gesamten Datenbestands des Legacy-Systems angelegt\footnote{Ein \textit{Dump} (Auszug) ist ein Abbild des Speicherzustandes etwa einer Datenbank zu einem bestimmten Zeitpunkt}. Dieser wird anschließend mittels der hierfür entwickelten Migrationsskripte in das neue System übertragen \citep[S.~3]{brodie-1993}. Bis die Datenmigration abgeschlossen ist, kann weder das neue noch das Legacy-System im Unternehmen eingesetzt werden. Daraus resultiert ein Zeitraum, in welchem das Unternehmen ohne ein lauff"ahiges System auskommen muss und somit stark in seiner Geschäftsfähigkeit eingeschränkt ist \citep[S.~3f.]{brodie-1993}.
\newpage
Die Vorteile diese Ansatzes beschränken sich auf das verbesserte Programmverständnis, die Performance und die bessere Wartbarkeit des neu entwickelten Systems, nachteilig hingegen ist ein hohes Risiko bei Fehlschlag der Einf"uhrung \citep[S.~105]{bisbal-1999}. Das hohe Risiko des Big-Bang-Ansatz setzt sich aus mehreren Faktoren zusammen. Einer dieser Faktoren ist die Identifikation der Anforderungen an das neue System aus der Dokumentation und dem Programmcode des Legacy-System. Je länger das Legacy-System in Betrieb gewesen ist, desto wahrscheinlicher ist auch, dass dessen Dokumention veraltet ist. Einzelne Funktionalitäten und Abhängigkeiten innerhalb des Legacy-Systems könnten falsch oder nur unzureichend dokumentiert sein \citep[S.~2]{brodie-1993}. Darüber hinaus bestehen auch übliche Probleme eines Softwareprojekts. Anforderungen können sich im Laufe der Entwicklung ändern, Fristen können überzogen werden und die Kosten für die Entwicklung des neuen Systems in die Höhe treiben \citep[S.~2f.]{brodie-1993}. Neben dem hohen Risiko besteht bei Nutzung Big-Bang-Ansatzes allerdings auch die zeitweise Unterbrechung aller vom Legacy- bzw. dem neuen System unterstützten Geschäftsprozesse \citep[S.~4]{wuLawless-1997}.
\lb
Zusammenfassend bietet der Big-Bang-Ansatz auf der einen Seite eine unkomplizierte Datenmigration. Auf der anderen Seite jedoch stehen ein hohes Risiko bei der Einf"uhrung sowie Ausfallzeiten während der Einf"uhrung der "Anderungen. In diesem Zeitraum k"onnen Gesch"aftst"atigkeiten nur eingeschr"ankt und in keinster Weise durchgef"uhrt werden. Der Big-Bang-Ansatz kann somit nicht in jeden Unternehmen, sondern nur in einige wenigen eingesetzt werden. 

\subsection{Chicken-Little-Ansatz}

%TODO Hier noch Grafik einbauen

Der Chicken-Little-Ansatz stellt eine weitere Vorgehensweise zur Datenmigration dar. Entwickelt wurde dieser Ansatz von Michaell Brodie und Michael Stonebraker im Rahmen des 1991 begonnenen DARWIN-Projekts der University of California in Berkeley \citep{zoulafy-2002}. Der Chicken-Little-Ansatz stellt im Gegensatz zum Big-Bang-Ansatz ein inkrementelles Vorgehen zur Datenmigration dar. Durch das inkrementelle Vorgehen sollen Schwächen des Big-Bang-Ansatzes beseitigt werden. So können bei Chicken-Little-Ansatz beispielsweise Ausfallzeiten der von Legacy- bzw. neuen System unterstützten Geschäftsprozesse während der Datenmigration vermieden oder zumindest eingeschr"ankt werden \citep{zoulafy-2002}.
\lb
Das inkrementelle Vorgehen ermöglicht hierbei eine Entwicklung und Einf"uhrung aller "Anderungen im Rahmen kleinerer Module. Entwickelt und eingef"uhrt werden zun"achst wenige Anpassungen. Inkrementell k"onnen weitere Funktionalitäten des Legacy-System auf übertragen werden \citep[S.~2]{wuLawless-1997} .
\lb
Um auch während der Datenmigration die kontinuierliche Arbeit mit Softwaresystemen zu ermöglichen, kommt beim Chicken-Little-Ansatz ein sogenanntes Gateway zum Einsatz. Dieses Gateway verbindet Legacy- und das neue System nach "Anderung oder Neuentwicklung miteinander und koordiniert deren Kommunikation untereinander. Somit können das neue und das Legacy-System solange parallel ausgeführt werden, bis das neue System alle Daten enth"alt und alle Funktionalitäten übernehmen kann \citep[S.~2]{wuLawless-1997}. Die neuen Daten, die während der Zeit entstehen in der das neue System noch nicht alle Funktionen übernommen hat, werden dabei auf dem für die Funktion zuständigen System gespeichert. Folglich werden Daten, welche durch bereits ans neue System übertragene Funktionen erstellt worden sind auch dort gespeichert und nicht mehr auf dem Legacy-System \citep[S.~2]{wuLawless-1997}. In gewissen Kontexten ist es sinnvoll, Daten zeitweise in beiden Systemen zu pflegen. Urspr"ungliche Schemata und Datenquellen werden erst verworfen, sobald zuk"unftige System diese Aufgaben verl"asslich erf"ullen k"onnen.
\lb
Das Gateway erfüllt nun zwei Aufgaben beim Übertragen der Daten an das neue System. Zum einen ist dies das Bereitstellen der noch nicht migrierten Daten an das neue System und zum Anderen das Bereitstellen der bereits migrierten Daten an das Legacy-System. Ersteres wird auch als umgekehrtes Gateway (engl.: \textit{reverse Gateway}) und letzteres als vorwärts Gateway (engl.: \textit{forward Gateway}) bezeichnet \citep[S.~2]{wuLawless-1997}. Je nachdem, wie stark sich Datenbankschemata von Legacy- und neuen System unterscheiden, stellen vorwärts und umgekehrtes Gateway unterschiedlich komplexe Module dar. Mit steigender Komplexität dieser Module sinkt allerdings auch die Performanz des System während der Migrationsphase \citep[S.~109]{bisbal-1999}.
\lb
"Ahnlich dem Big-Bang-Ansatz, bietet der Chicken-Little-Ansatz die erwarteten Verbesserungen in Performance, Wartbarkeit sowie im Verständnis des Systems \citep[S.~108]{bisbal-1999}. Allerdings fallen bei Einsatz des Chicken-Little-Ansatzes keine Ausfallzeiten, in denen weder mit Legacy- noch mit dem neuen System gearbeitet werden kann, an. Ermöglicht wird dies durch den Einsatz des Gateways, wodurch auch während der Migrationsphase mit dem gesamten System gearbeitet werden kann \citep[S.~2]{wuLawless-1997}. Darüber hinaus bietet der Chicken-Little-Ansatz noch weitere Vorteile. Es muss beispielsweise nicht das gesamte Legacy-System auf einmal ersetzt werden. Stattdessen kann dieses Modul f"ur Modul erneuert werden. Dies erleichtert Entwicklung des neuen Systems und damit allen Anpassungen \citep[S.~3]{brodie-1993}. Die inkrementelle Anpassung des neuen Systems bietet  auch Vorteile in der Fehlerbehandlung. So müssen bei Auftreten von Fehlern nur  entsprechende Schritte wiederholt werden. Wenn etwa ein Problem bei der Datenmigration eines entwickelten Moduls auftritt, muss nach der Fehlerbehebung lediglich die Datenmigration von diesen Moduls wiederholt werden. Durch die Fehlerbehebung gewonnene Kenntnisse können ebenfalls in den nächsten Schritten genutzt werden, um bekannte Probleme zu vermeiden, bevor diese entstehen \citep[S.~3]{brodie-1993} .
\lb
% je nach LIS unterschiedlich schwer durchzuführen
Neben den genannten Vorteilen hat der Chicken-Little-Ansatz auch einige Nachteile aufzuweisen. Zwar erlaubt der Einsatz eines Gateways während der Migrationsphase mit einem kombinierten System zu arbeiten, dafür stellt die Entwicklung des Gateway eine Herausforderung dar. Sie zieht unter Umst"anden hohe Kosten dar unter erfordert eine hohes fachliches Verst"andnis beider Systeme. Daten, welche auf beiden Systemen vorhanden sind, müssen konsistent gehalten werden. Um die Interoperabilität der beiden Systeme zu ermöglichen, muss das Gateway semantische und technische Verkn"upfungen beider Schemata herstellen. Je unterschiedlicher die Datenbankschemata von Legacy- und neuem System sind, desto komplexer erscheint diese Aufgabe \citep[S.~2f]{wuLawless-1997}. Darüber hinaus muss bei komplexeren Gateways auch mit Einbussen im Hinblick auf Performance gerechnet werden, wenn ein vorw"arts- beziehungsweise umgekehrtes Gateway zum Einsatz kommt \citep[S.~109]{bisbal-1999}.
\lb
Alles in allem betrachtet bietet der Chicken-Little-Ansatz viele Vorteile. Die inkrementelle Herangehensweise minimiert viele Risiken und Probleme, wie z.B. die Ausfallzeiten des Big-Bang-Ansatzes. Aus diesem Grund eignet sich Chicken-Little auch zum Einsatz größeren Unternehmen. Dennoch muss bedacht werden, dass die Durchf"uhrung des Chicken-Little-Ansatzes auch schnell sehr komplex werden kann.

\subsection{Butterfly-Ansatz}


% versucht Problematik des Gateway zu umgehen
% Daten des Legacy-System werden in TempStorages übertragen
% Legacy-System bleibt bis Abschluss im Betrieb
% Nach Übertragung der initialen Daten mittels temp0, wird temp1 mit den neuen Daten angelegt ...
% temp werden mit der Zeit immer kleiner bis temp so klein ist das dieses ausreichend schnell übertragen werden kann -> Abschalten des Legacy-System und abschließende Übertragung
% minimal Downtime für finale Datenübertragung bis das neue System eingesetzt werden kann

Im Rahmen des MILESTONE-Projekt, welches 1996 startete, wurde in einer Kooperation vom Trinity College Dublin, Broadcom \'{E}ireann Research, Telecom \'{E}ireann und Ericson der sogenannte Butterfly-Ansatz entwickelt.\citep[S. 202]{wuLawlessBisbal-1997} Im Gegensatz zum Big-Bang- und Chicken-Little-Ansatz trennt der Butterfly-Ansatz die Entwicklung bzw. Anpassung eines neuen Systems explizit von der Datenmigration. Durch diese Trennung benötigt der Butterfly-Ansatz keine Gateways, die zwischen den Systemen vermitteln.\citep[S. 202]{wuLawlessBisbal-1997} Somit ist es auch nicht Möglich, dass während der Datenmigration auf beiden Systemen gearbeitet werden kann. Allerdings ist diese Aussage nicht mit einer längeren Ausfallzeit, wie beim Big-Bang-Ansatz, gleichzusetzen. Stattdessen wird beim Butterfly-Ansatz mit temporären Datenspeichern gearbeitet, die die weiterarbeitet am Legacy-System während der Datenmigration ermöglichen. Das neue System kommt erst zum Einsatz wenn alle Daten auf dieses übertragen worden sind, bis dies geschehen ist bleibt das Legacy-System im Einsatz.\citep[S. 3]{wuLawless-1997}
\lb
% Grafik FIGURE 3 MIGRATING DATA IN TEMPSTORE TSn ( BUTTERFLY METHODOLOGY )

Zu Beginn der Datenmigration wird beim Butterfly-Ansatz der aktuelle Datenbestand des Legacy-Systems enigefroren. Ab diesem Zeitpunkt können auf dem Legacy Datenbestand nur noch Lese-Aktionen durchgeführt werden, jedoch können dort keine neuen Daten mehr hinterlegt werden.\citep[S. 202]{wuLawlessBisbal-1997} Um dennoch mit dem Legacy-System weiterarbeiten zu können bedient sich der Butterfly-Ansatz eines temporären Speichers sowie eines Data-Access-Allocator. Dieser Data-Access-Allocator enthält Informationen darüber wo welche Daten im eingefrorenen Datenbestand zu finden sind und welcher temporärer Speicher gerade aktiv ist.\citep[S. 202]{wuLawlessBisbal-1997} Wie in Grafik ? zu sehen ist wird der Data-Access-Allocator zwischen die Anwendungen und Dienste des Legacy-System und deren Datenbank geschaltet. Werden nun neue Daten ins Legacy-System eingegeben oder werden vorhandene bearbeitet sorgt der Data-Access-Allocator dafür, dass das die entsprechenden Daten aus dem eingefrorenen Datenbestand gelesen werden und neue oder im aktuellen temporären Speicher abgelegt werden.\citep[S. 202]{wuLawlessBisbal-1997}
\lb
Durch Data-Access-Allocator und temporäre Speicher wird sichergestellt, dass während der Datenmigration mit dem Legacy-System normal weitergearbeitet werden kann. Für die Übertragung der Daten auf das neue System wird allerdings noch ein Daten Transformator, ein sogenannter Chrysaliser benötigt.\citep[S. 202]{wuLawlessBisbal-1997} Dieser Chrysaliser übernimmt die Rolle der Migrationsskripte. Er enthält die Informationen über die Datenbankschemata des Legacy- und des neuen Systems und wie die des Legacy-System umgewandelt werden müssen.\citep[S.202]{wuLawlessBisbal-1997} 
\lb
Mit Hilfe des Chrysaliser werden nun die Daten aus dem eingefrorenen Datenbestand an das neue System übertragen. Sobald alle Daten aus dem eingefrorenen Datenbestand übertragen wurden, wird ein weiterer temporärer Speicher t2 angelegt und der Datenbestand des alte temporäre Speicher eingefroren. Anschließend muss der Data-Access-Allocator entsprechend angepasst werden.\citep[S. 202]{wuLawlessBisbal-1997} Der Chrysaliser kann nun damit beginnen den temporären Speicher t1 auf das neue System zu übertragen. Ist auch t1 übertragen wird t2 eingefroren und ein temporärer Speicher t3 für neuen im Legacy-System angelegte Daten angelegt.\citep[S. 202]{wuLawlessBisbal-1997} Mit jeder neuen Iteration des temporären Speichers reduziert sich dessen Größe. Dieser Umstand lässt sich darauf zurückführen, dass in der Zeit die für die Migration des initialen Datenbestands benötigt wird sich in der Regel der Datenbestand nicht verdoppelt. Der temporäre Speicher t1 müsste somit auch kleiner sein als der initiale Datenbestand des Legacy-Systems. Analoges gilt für die folgenden Iterationen des temporären Speichers: \textbf{$t_n <t_{n+1}$}.\citep[S. 202]{wuLawlessBisbal-1997} 
\lb
Dieser Vorgang wird solange wiederholt bis ein der temporäre Speicher eine zu Beginn der Datenmigration festgelegte Größe erreicht. Wenn der temporäre Speicher diese Größe nun erreicht und es an der Zeit ist diesen durch den Chrysaliser auf das neue System zu migrieren, wird dieser wie auch schon seine Vorgänger eingefroren allerdings wird kein neuer temporärer Speicher eingerichtet. Stattdessen wird ab diesem Zeitpunkt der Betrieb auf dem Legacy-System eingestellt damit die letzten Daten migriert werden können.\citep[S. 202]{wuLawlessBisbal-1997} Aus diesem Grund sollte die Größe des letzten temporären Speichers so gewählt werden, dass eine schnelle migration von diesem möglich ist damit die Ausfallzeit des System minimal gehalten werden kann.\citep[S. 202]{wuLawlessBisbal-1997} Nachdem die Datenmigration nun abgeschlossen ist kann das neue System in Betrieb genommen und das Legacy-System endgültig abgeschaltet werden.\citep[S. 204]{wuLawlessBisbal-1997}
\lb
Neben den allgemeinen Vorteil, wie z.B. verbesserte Wartbarkeit, bietet der Butterfly-Ansatz eine minimale Beeinträchtigung durch Systemausfälle in denen die vom System unterstützten Geschäftsprozesse nicht ausgeführt werden können. Durch die Wahl einer entsprechend kleinen Größe des letzten temporären Speichers kann die anfallende Ausfallzeit flexibel an die Bedürfnisse des Unternehmens angepasst werden.\citep[S. 204f]{wuLawlessBisbal-1997} Darüber hinaus kann beim Butterfly-Ansatz im Gegensatz zu anderen Ansätzen die für die Migration benötigte Gesamtzeit relativ sicher anhand des initialen Datenbestands des Legacy-Systems sowie Geschwindigkeit von Data-Access-Allocator und Chrysaliser eingeschätzt werdeen. Durch die Einschätzung des Zeitaufwands ist es dem Unternehmen möglich die Migration besser zu planen und so unnötige Problematiken zu vermeiden. Beispielsweise kann verhindert werden, dass die Migration des System in eine Zeit fällt in der das System stärker als normal beansprucht wird.\cite[S. 204]{wuLawlessBisbal-1997} Des Weiteren entfällt beim Butterfly-Ansatz die Notwendigkeit zur Entwicklung eines Gateway um während der Datenmigration weiterarbeiten zu können, da das Legacy-System bis zum Abschluss im Betrieb bleib. Das bietet ebenfalls den Vorteil, dass keine Probleme mit der Datenkonsitenz auftreten können.\citep[S. 3]{wuLawless-1997}
\lb
Allerdings hat der Butterfly-Ansatz auch Nachteile, die bei der Entscheidung welche Vorgehensweise verwendet werden soll berücksichtigt werden müssen. So muss zwar kein komplexes Gateway entwickelt werden, dafür aber ein Data-Access-Allocator, der temporäre Speicher einrichtet und die Anfragen aus dem Legacy-System entsprechend umleiten kann sowie ein Chrysaliser der die Daten auf das neue System übertragen kann.\citep[S. 3]{wuLawless-1997} Sowohl Data-Access-Allocator als auch Chrysaliser sind entscheidend daür wie schnell und sicher eine Migration mit dem Butterfly-Ansatz durchgeführt werden kann. Aus diesem Grund wird auch hier ein hohes technisches Verständnis für die Entwicklung benötigt.\citep[S. 204]{wuLawlessBisbal-1997} Des Weiteren kann es hier passieren, dass je nach zu migrierenden Legacy-System ein große Anzahl an temporären Speichern benötigt wird. Folglich kann es zu einem hohen Hardwarebedarf für Speicher kommen, wodurch wiederum die Kosten der Migration in die Höhe getrieben werden.\citep[S. 109f]{bisbal-1999}
\lb
Zusammenfassend betrachtet stellt der Butterfly-Ansatz wie auch schon Big-Bang und Chicken-Little keine allgemeingültige Lösung dar. Auch hier muss beachtet werden das der Butterfly-Ansatz sich nicht zur Migration mit jedem Legacy-System eignet. 
