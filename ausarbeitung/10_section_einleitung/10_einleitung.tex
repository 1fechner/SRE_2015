% !TeX spellcheck = de_DE
\section{Einleitung}
\label{chapter:einleitung}

%\huge
%\textbf{Aufgabenstellung: Bewerten Sie Ansätze der Datenmigration bezüglich Stärken, Schwächen, Einsatzbereich!} 
%\normalsize
%\lb

Die Datenmigration spielt im Kontext des Software-Reengineering eine entscheidende Rolle. Basis vieler Softwaresysteme bilden dynamische Daten in unterschiedlichen Formaten und auf Basis unterschiedlicher Technologien. Gesch"aftsprozesse basieren h"aufig auf bestimmten Datenstrukturen und -schemata. Als Migration bezeichnet man dabei das Verlagern von Daten auf andere Medien, das Konvertieren in neue Formate und die Restrukturierung bereits vorhandener Daten \citep{morris-2012}. Die Anpassung der umliegenden Software-Systeme spielt eine wichtige Rolle \cite{henrard-2002}.
\lb
Konkret werden im Kontext der Datenmigration vorhandene Daten f"ur das Reengineering von Software verwertet. Wichtige Kunden- oder Gesch"aftsdaten m"ussen auch nach Reengineering, Anpassung und Neuentwicklung von Softwaresystemen weiterhin genutzt werden. Ein strukturiertes Vorgehen bei der Analyse, Konvertierung, Portierung und Anpassung dieser Daten bildet die Grundlage der Datenmigration.  
\lb
Unterschiedliche Strategien versuchen auf verschiedenen Ebenen, die Migration von Daten zu strukturieren. Im Kontext des Reengineering von Daten m"ussen nicht alleine Datenquellen und -schemata angepasst werden. Im Zusammenhang mit urspr"unglichen Daten entwickelte Legacy-Systeme m"ussen neuen Datenquellen angepasst und entsprechend adaptiert werden \citep{henrard-2002}.	
\lb
F"ur Durchf"uhrung und Einf"uhrung der der Datenmigration im Kontext von Legacy-Systemen existieren weitere Ans"atze. Richtlinien bei der Durchf"uhrung von Migrationsprojekten geben das Vorgehen bei Durchf"uhrung dieser vor. Die Integration der Migration in Reengineering und Gesch"aftsprozesse spielte eine entscheidende Rolle f"ur die erfolgreiche Umsetzung der Datenmigration \citep{wuLawless-1997} \citep{ackermann-2005}.
\lb
Aus einer Vielfalt von Ans"atzen und Strategien ergeben sich je entsprechende Vor- und Nachteile. Schon eine grobe Analyse zeigt St"arken und Schw"achen der jeweiligen Methodiken. Unterschiedliche Ebenen f"ur die Anwendung der Strategien lassen sich auf verschiedene Anwendungsbereiche und Beteiligte zur"uckf"uhren.
\lb
Der Rest dieser Ausarbeitung ist wie folgt strukturiert: Kapitel \ref{chapter:motivation} beschreibt die Datenmigration im Kontext des Reengineering und gibt Hinweise auf die Wichtigkeit und Relevanz der Durchf"uhrung. Kapitel \ref{chapter:strategien} stellt Strategien der Datenmigration auf zwei Ebenen vor und beschreibt deren Vor- und Nachteile sowie deren Einsatzbereich. Kapitel \ref{chapter:richtlinien} gibt Richtlinien f"ur die Durchf"uhrung von Migrationsprojekten vor und Kapitel \ref{chapter:vorgehensweisen} beschreibt das Vorgehen bei der Einf"uhrung der Ergebnisse der Datenmigration. Abschlie"send gibt Kapitel \ref{chapter:fazit} eine b"undige Zusammenfassung der Inhalte.