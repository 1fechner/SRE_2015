% !TeX spellcheck = de_DE
\section{Generelles Vorgehen}
\label{chapter:richtlinien}
%TODO Julian
%-http://www.information-management.com/specialreports/20040518/1003611-1.html
%-http://searchsmbstorage.techtarget.com/tip/Data-migration-strategies-and-best-practices
%-https://pure.fundp.ac.be/ws/files/168599/wcre02.pdf
%http://www.dtic.upf.edu/~jbisbal/publications/datasem97.pdf
%http://dc-pubs.dbs.uni-leipzig.de/files/Rahm2000DataCleaningProblemsand.pdf
%http://csis.pace.edu/~marchese/CS775/Proj1/legacyinfosys_directions.pdf
%http://www.oracle.com/technetwork/middleware/oedq/successful-data-migration-wp-1555708.pdf




%\section{Generelles Vorgehen}

%Unabh"angig von der gew"ahlten Strategie setzt die Durchf"uhrung einer Datenmigration drei grunds"atzliche T"atigkeiten voraus \citep{henrard-2002}. Diese dienen als Ger"ust der Umsetzung einer Migration. 

Die Datenmigration stellt einen risikoreichen Aufgabenbereich dar. Aus diesem Grund sollte diese nicht ohne eine vorhergehende Planungsphase und ohne Struktur durchgeführt werden. Ein allgemeingültiges Verfahren mit dem das Risiko der Datenmigration reduziert werden existiert jedoch nicht\citep[S.~3]{wuLawless-1997}. Dennoch gibt es verschiedene Ansätze um die den Prozess der Migration zu strukturieren. Einer dieser Ansätze ist das von Klaus Haller, Florian Matthes und Christopher Schulz in Zusammenarbeit mit Unternehmen aus Automobil-Industrie und dem Finanzsektor entwickelte Prozessmodell für Datenmigration\citep[S.~2f.]{klausMatthesSchulz-2012}. 
\lb
Dieses Prozessmodel unterteilt die Migration in vier Abschnitte: Die Initialisierung, die Migrationsentwicklung, dem Testabschnitt und die Umstellung auf das neue System. Für jeden dieser Abschnitte werden dabei die wichtigtes Tätigkeiten und Ergebnisse dargestellt\citep[S.~5f]{klausMatthesSchulz-2012}. Je nach Unternehmen und der Migrationssituation können die Tätigkeiten etwas variieren.
\lb
In der Initialisierung muss die aktuelle Situation analysiert werden. Hierbei gilt es herauszufinden um was für ein Legacy-System es sich handelt, wo und wie die Daten gespeichert sind, auf was für ein System migriert werden soll und welche Daten mitgenommen werden sollen.\citep[S.~7]{klausMatthesSchulz-2012} Auf dieser Grundlage wird die Entscheidung darüber getroffen welche Vorgehensweise für Migration verwendet werden soll. Diese Information sind somit essentiell um das weitere Vorgehen planen zu können und eine initiale Aufwandseinschätzung zu erstellen.\citep[S.~7]{klausMatthesSchulz-2012} Ebenfalls wird auf dieser Grundlage der gesammtelten Informationen eine Migrationsplattform eingerichtet. Mit Hilfe dieser Migrationsplattform werden im folgendem die benötigten Migrationswerkzeuge, -skripte usw. entwickelt und getestet.\citep[S.~7]{klausMatthesSchulz-2012}
\lb
Im Abschnitt zwei, der Migrationsentwicklung, wird zunächst die vorangegange Analyse der auf dem Legacy-System vorhandenen Daten vertieft. Dafür wird ein Backup des Legacy-Datenbestands auf die Migrationsplatform überspielt. Somit kann verhindert werden, dass während der Analyse Probleme oder Störungen auf dem Legacy-System auftreten.\citep[S.~7]{klausMatthesSchulz-2012} Bei der Analyse ist festzustellen welche Datenschemata und -formate auf dem Legacy-System und auf dem neuen System vorhanden sind. Besonders wichtig ist hierbei worin sich diese voneinander unterscheiden. Dieses Wissen ermöglicht es Migrationsskripte zu entwickeln, mit denen die Daten automatisiert konvertiert und übertragen werden können\citep[S.~7f.]{klausMatthesSchulz-2012}. 
\lb
Nachdem die Struktur der Daten analysiert worden, müssen die Daten selbst betrachtet werden. Man versucht auf diese Weise unvollständige, inkonsistente oder doppelte Datensätze zu identifizieren. Um Probleme im neuen System zu vermeiden sollten solche Datensätze bereinigt werden. Die Bereinigung(engl.: Cleansing) kann entweder vor der Datenmigration auf dem Legacy-System durchgeführt werden, während mit Hilfe der Migrationsskripte oder im Anschluss auf dem neuen System.\citep[S~7f.]{klausMatthesSchulz-2012}. Mittels dieser Bereinigung werden die inkonsisten und unvollständigen Datensätze repariert und die Doppelten entfernt\citep[S.~7f.]{rahm-2010}. 
\lb
Die in der Migrationsentwicklung entwickelten müssen anschließend im Testabschnitt, ausgiebig auf ihre Funktionalität überprüft werden. Entsprechende Test werden auch hier auf der Migrationsplattform ausgeführt\citep[S.~8f]{klausMatthesSchulz-2012}. Gefundene Fehler werden korrigiert und anschließend werden die Migrationsskripte erneut überprüft. Dieser Prozess wird solange wiederholt bis die Migrationsskripte fehlerfrei sind und deren Ergebnisse als korrekt validiert sind\citep[S.~8f]{klausMatthesSchulz-2012}
\lb
Abschließend folgt die Umstellung auf das neue System. Dieser Abschnitt beinhaltet die Migration vom Legacy- auf das neue System mit Hilfe der zuvor entwickelten Migrationsskripte. Je nach der gewählten Vorgehensweise bleibt das Legacy-System bis zum Abschluss der Migration und Sicherstellung der Funktionalität des neuen System im Einsatz\citep[S.~107]{bisbal-1999}. 
\lb
Nachdem die Datenmigration abgeschlossen wurde, ist es in der Regel sinnvoll noch einen Erfahrungsbericht zu erstellen. In diesem soll festhalten werden, in welchen Phasen der Migration es zu Probleme kam und wie diese behoben wurden. Auf diese Weise soll der Migrationsprozess im Unternehmen aktiv verbessert werden. Ebenfalls soll sichergestellt werden, dass in zuk"unftigen Migrationsprojekten nicht dieselben Probleme auftreten \citep[S.~10]{klausMatthesSchulz-2012}.


