% !TeX spellcheck = de_DE
\section{Allgemeine Planungsrichtlinien}
%TODO Julian
%-http://www.information-management.com/specialreports/20040518/1003611-1.html
%-http://searchsmbstorage.techtarget.com/tip/Data-migration-strategies-and-best-practices
%-https://pure.fundp.ac.be/ws/files/168599/wcre02.pdf
%http://www.dtic.upf.edu/~jbisbal/publications/datasem97.pdf
%http://dc-pubs.dbs.uni-leipzig.de/files/Rahm2000DataCleaningProblemsand.pdf
%http://csis.pace.edu/~marchese/CS775/Proj1/legacyinfosys_directions.pdf
%http://www.oracle.com/technetwork/middleware/oedq/successful-data-migration-wp-1555708.pdf

Die Daten Migration stellt einen risikoreichen Aufgabenbereich dar. Aus diesem Grund sollte diese auch nicht ohne eine vorhergehende Planungsphase durchgeführt werden. Allerdings keine Standardvorgehensweise für diese Planungsphase.\cite[vgl.][S. 3]{wuLawless-1997} Stattdessen gibt es verschiedene Sammlungen von allgemeinen Richtlinien zur erfolgreichen Durchführung der Datenmigration. In folgendem werden ein paar dieser Richtlinie besprochen.

\subsection{MILESTONE-Richtlinien}

% Hier Grafik Übersicht MILESTONE

Das MILESTONE-Projekt stellte eine Kollaboration des Trinity College Dublin, Broadcom Éireann Research, Telecom Éireann und Ericssons dar. Es begann 1996 und endete 1998. Im Zuge dieses Projektes sollten Migrations-Methoden und diese Methoden unterstützende Werkzeuge entwickelt werden.\cite[vgl.][S. 3]{wuLawless-1997} Unter anderen wurde im MILESTONE-Projekt auch die Butterfly-Vorgehensweise zur Datenmigration entwickelt, die in Kapitel 5 besprochen wird.\cite[vgl.][S. 3]{wuLawless-1997} 

Im MILESTONE-Projekt unterteilt man den Migrationsprozess in fünf Aufgabenbereiche: Rechtfertigung(engl.: Justification), Verstehen des Legacy-Systems(engl.: Legacy System Understanding), Entwicklung des Zielsystems(engl.: Target System Development), Migration und Testen.\cite[vgl.][S. 3]{wuLawless-1997} 


\subsection{4-Phasen-Modell von Haller, Matthes und Schulz}