% !TeX spellcheck = de_DE
\section{Allgemeine Planungsrichtlinien}
\label{chapter:richtlinien}
%TODO Julian
%-http://www.information-management.com/specialreports/20040518/1003611-1.html
%-http://searchsmbstorage.techtarget.com/tip/Data-migration-strategies-and-best-practices
%-https://pure.fundp.ac.be/ws/files/168599/wcre02.pdf
%http://www.dtic.upf.edu/~jbisbal/publications/datasem97.pdf
%http://dc-pubs.dbs.uni-leipzig.de/files/Rahm2000DataCleaningProblemsand.pdf
%http://csis.pace.edu/~marchese/CS775/Proj1/legacyinfosys_directions.pdf
%http://www.oracle.com/technetwork/middleware/oedq/successful-data-migration-wp-1555708.pdf

Die Datenmigration stellt einen risikoreichen Aufgabenbereich dar. Aus diesem Grund sollte diese auch nicht ohne eine vorhergehende Planungsphase durchgeführt werden. Allerdings keine Standardvorgehensweise für diese Planungsphase.\cite[S. 3]{wuLawless-1997} Stattdessen gibt es verschiedene Sammlungen von allgemeinen Richtlinien zur erfolgreichen Durchführung der Datenmigration. In folgendem werden ein paar dieser Richtlinie besprochen.

\subsection{MILESTONE-Richtlinien}

% Hier Grafik Übersicht MILESTONE

Das MILESTONE-Projekt stellte eine Kollaboration des Trinity College Dublin, Broadcom Éireann Research, Telecom Éireann und Ericssons dar. Es begann 1996 und endete 1998. Im Zuge dieses Projektes sollten Migrations-Methoden und diese Methoden unterstützende Werkzeuge entwickelt werden.\cite[S. 3]{wuLawless-1997} Unter anderen wurde im MILESTONE-Projekt auch die Butterfly-Vorgehensweise zur Datenmigration entwickelt, die in Kapitel 5 besprochen wird.\cite[S. 3]{wuLawless-1997} 
\lb
Im MILESTONE-Projekt unterteilt man den Migrationsprozess in fünf Aufgabenbereiche: Rechtfertigung(engl.: Justification), Verstehen des Legacy-Systems(engl.: Legacy System Understanding), Entwicklung des Zielsystems(engl.: Target System Development), Migration und Testen.\cite[S. 3]{wuLawless-1997} Jede dieser Aufgabenbereiche stellt eine notwendiges Erfordernis für den Erfolg der Migration dar. So wird innerhalb des Aufgabenbereiches der Rechtfertigung die Fragestellung geklärt, ob die Entwicklung eines neuen Systems sinnvoll ist oder ob noch weiter mit dem Alten weitergearbeitet werden soll. Um diese Fragestellung zu klären, müssen zunächst die Vorteile und auch die Risiken des Legacy System sowie des potentiell neu zu entwickelnden System gesammelt und einander gegenüber gestellt werden.\cite[S. 3]{wuLawless-1997}
\lb
Wenn geklärt ist ob der Einsatz eines neuen Systems gerechtfertigt ist, kann mit den weiteren Aufgabenbereichen begonnen werden. Zum verstehen des Legacy-Systems muss dieses und dessen zugehörige Dokumentation, sofern vorhanden, analysiert werden.\cite[S. 3f]{wuLawless-1997} Hierbei ist muss geklärt werden was für Daten auf den Legacy-System und wie diese gespeichert sind. Dafür kommen Techniken des Reverse Engineering zum Einsatz.\cite[S. 3f]{wuLawless-1997} In dieser Phase müssen alle Funktionen und die vom System zu deren Ausführung benötigten Daten identifiziert werden. Des Weiteren muss der Datenbestand des Legacy-System auf redundante, unvollständige und veraltete Datensätze untersucht werden.\cite[S. 3f]{wuLawless-1997}
\lb
Sobald nun alle Daten und Funktionalitäten des Legacy-Systems identifiziert wurden, kann mit der nächsten Aufgabenbereich, der Entwicklung des Zielstystems, begonnen werden. Ziel ist es hier ein neues System zu entwickeln, welches alle Funktionen des Legacy-Systems abdeckt. Darüber hinaus sollen im neuen System jedoch Nachteile des Legacy-Systems, wie z.B. schlechte Wartbarkeit, beseitigt werden.\cite[S. 3ff]{wuLawless-1997} 
\lb
Im nächsten Aufgabenbereich der Migration nach MILESTONE findet die Migration statt. Abhängig von der gewählten Vorgehensweise verläuft dieser Aufgabenbereich unterschiedlich. Prinzipiell soll hier aber der Umzug vom Legacy- auf das neue System durchgeführt werden. Das beinhaltet auch die Migration der Daten und der Anwendungen, aber auch das Einarbeiten der Benutuer auf dem neuen System.\cite[S. 3ff]{wuLawless-1997}
\lb
Anderes als die vorherigen Aufgabenbereiche, die sich nur durch ihre Ergebnisse miteinander interagieren, überwacht das Testen sämtliche Aktivitäten der einzelnen Phasen jenseits der Rechtfertigung. Durch die rechtzeitige Durchführung Tests in jeder Aufgabenbereiche sollen mögliche Fehler und Ungenauigkeit identifiziert  und behoben werden.\cite[S. 3f]{wuLawless-1997}

% passendes Ende noch ergänzen

\subsection{4-Phasen-Modell von Haller, Matthes und Schulz}

% Hier noch Grafik einfügen
% Stages statt Phasen!!!

Das 4-Phasen-Modell von Haller, Matthes und Schulz wurde in Zusammenarbeit mit mehreren Unternehmen aus der Automobil-Industrie, dem Finanzsektor sowie weiteren aus anderen Sektoren entwicklet.\citep[S. 2f]{klausMatthesSchulz-2012} Es soll dazu dienen den Prozess der Datenmigration zu modellieren. Hierbei steht der Augenmerk von den Autoren auf der Einteilung des Migrationsprozesses in vier Phasen sowie deren einzelne Aufgaben, Zuständigkeiten und Ergebnissen.\citep[S. 5f]{klausMatthesSchulz-2012}
\lb
Phase eins wurde von den Autoren als Initialisation betitelt. In dieser Phase wird zunächst geklärt wer die bevorstehende Datenmigration durchführen soll. Zur Auswahl stehen hier die hauseigene IT-Abteilung oder ein externer IT-Dienstleister.\citep[S. 7]{klausMatthesSchulz-2012} Nachdem die Entscheidung getroffen wurde wer für die Migration zuständig ist, kann mit der ersten Analyse des bestehenden Legacy-Systems begonnen werden. Herausgefunden soll bei dieser ersten Analyse welche Daten vorhanden sind, welche der bestehenden Daten migriert werden sollen und welche verworfen werden können.\citep[S. 7]{klausMatthesSchulz-2012} Des Weiteren müssen vorhandene organisatorische und technische Einschränkungen identifiziert werden, wie z.B eine bereits bestehende Datenbank die ins neue System integriert werden soll oder das Vorhandensein von an mehereren Standorten verteilten Teams. Auch muss festgelegt werden welche Abnahmekriterien das neue System erfüllen muss.\citep[S. 7]{klausMatthesSchulz-2012} Auf dieser Grundlage muss anschließend auch die Vorgehensweise der Datenmigration festgelegt werden.\citep[S. 7]{klausMatthesSchulz-2012} Ein dritter Aufgabenbereich der in der Initialisation Phase abgearbeitet werden muss ist das aufsetzen der Datenmigrationsplattform. Diese Datenmigrationsplattform dient im folgenden dem Datenmigrationsteam zum entwicklen und testen der benötigten Werkzeuge, wie z.B. den Migration-Skripten. Darüber bildet die Datenmigrationsplattform aber auch die Grundlage für die Testmigrationen.\citep[S.7]{klausMatthesSchulz-2012}
\lb
In der Phase zwei, der Migrationsentwicklung(engl.: Migration Development) wird zunächst eine Kopie des Aktuellen Datenbestands des Legacy-Systems auf der Migrationsplattform angelegt. Mittels dieser Kopie können die Migrationsskripte auf der Migrationsplattform entwickelt und getestet werden, ohne das Einschränkungen oder Probleme für die Ablaufe im Unternehmen entstehen.\citep[S. 7]{klausMatthesSchulz-2012} Im Anschluss daran müssen die Daten und die Struktur des Legacy-Systems eingehender untersucht werden. Während sich in Phase eins nur ein grober Überblick darüber verschafft wurde, welche Daten im Legacy-System vorhanden sind, muss jetzt die Struktur der Daten und deren Qualität betrachtet werden.\citep[S. 7f]{klausMatthesSchulz-2012} Bei der Analyse der Struktur muss überprüft werden wie die Daten abgespeichert wurden. Es wird erfasst welche Tabellen in der Datenbank existieren und über welche Attribute diese verfügen. Die hier gewonnenen Erkenntnisse müssen anschließend mit dem neuen System abgeglichen werden, damit auf die Abweichungen während der Entwicklung eingegangen und somit spätere Fehler vermieden werden können.\citep[S. 8]{klausMatthesSchulz-2012} Durch die Analyse der Datenqualität werden unvollständige oder fehlende Daten, die zunächst repariert bzw. ergänzt werden müssen identifiziert. Dies kann beispielsweise durch ergänzen zusätzlicher Informationen oder durch löschen der beschädigten Daten während oder vor der Migration erfolgen.\citep[S. 7f]{klausMatthesSchulz-2012} Als letztes findet in dieser Phase die Daten Transformation(engl.: Data transformation) statt. Ziel ist es hierbei die Transformationsregel für die Migrationsskripte aus den gewonnenen Erkenntnissen der Struktur- und Qualitätsanalyse zu generieren. In diesen Transformationsregel wird festgelegt wo welche Inhalte der Datenbanktabellen bzw. deren Attribute des Legacy-System im neuen System hinterlegt werden sollen.\citep[S. 8]{klausMatthesSchulz-2012}
\lb
Nachdem die Migrationsskripte in Phase zwei entwickelt worden sind, können diese nun in Phase drei, der Testphase(eng.: Testing) ausführlich getestet werden. Neben einen Migrationstestlauf auf der Migrationsplattform werden noch zusätzlich eine Erscheinungstest(engl.: Appearence test), ein Vollständigkeits- und Typen Übereinstimmungstest(engl.: Completeness \& type correspondent test), eine Verarbeitungstest(engl.: Processability test) sowie ein Integrationstest mit denen die Testmigration validiert werden soll.\citep[S. 8f]{klausMatthesSchulz-2012} So wird mittels des Erscheinungstest überprüft ob die Daten, abgesehen von den durch das neue System optischen Veränderungen, auch richtig angezeigt werden.\citep[S. 8f]{klausMatthesSchulz-2012} Der Vollständigkeits- und Typ Übereinstimmungstest soll herausfinden ob irgendwelche Daten während der Migration verloren gegangen sind oder sich nicht mehr denn richtigen Objekttyp, wie z.B. Kunde, zugeordnet werden kann.\citep[S. 9]{klausMatthesSchulz-2012} Verarbeitungs- und Integrationstest testen ob die migrierten Daten auch richtig vom neuem System verarbeitet werden können und ob das neue System auch mit den anderen Systemen des Unternehmen interagieren kann.\citep[S. 9]{klausMatthesSchulz-2012} Bei auftreten von Fehlern, muss deren Ursprung festgestellt werden und in den Migrationsskripten behoben werden. Die behoben Migrationsskripte müssen anschließend wieder erneut getestet werden. Erst wenn alle Tests erfolgreich durchlaufen wurde kann die Generalprobe(engl.: Final rehearsal) der Datenmigration beginnen. Hierbei wird erneut eine Testmigration durchgeführt, diesmal jedoch unter denselben Umständen die auch bei der richtigen Datenmigration gelten werden. Treten auch hier keine weiteren Auffälligkeiten mehr auf gilt die Testphase als erfolgreich abgeschlossen.\citep[S. 9f]{klausMatthesSchulz-2012}
\lb
Die letzte Phase des 4-Phasen-Modells von Haller, Matthes stellt den Übergang von Legacy- hin zum neuen System dar. In dieser Phase wird nicht mehr mit der Migrationsplattform gearbeitet, stattdessen werden die Migrationsskripte direkt auf das Legacy-System angewangt um die Daten auf das neue System zu migrieren.\citep[S. 10]{klausMatthesSchulz-2012} Je nach der gewählten Vorgehensweise kann das Legacy-System bis zum Abschluss der Migration noch aktiv im Unternehmen genutzt werden. 
% Belegendes Zitat und vllt. Ausführung?
Nachdem die Datenmigration abgeschlossen ist werden in der Regel noch ein Erfahrungsbericht erstellt. In diesem Erfahrungsbericht soll festhalten werden wo es Probleme gab und wie diese behoben wurde. Auf diese Weise soll der Migrationprozess im Unternehmen aktiv verbessert werden und sichergestellt werden, dass beim nächsten Mal nicht erneut dieselben Probleme auftreten.\citep[S. 10]{klausMatthesSchulz-2012}

% passendes Ende noch ergänzen