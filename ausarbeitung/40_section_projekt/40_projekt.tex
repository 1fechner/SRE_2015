% !TeX spellcheck = de_DE
\section{Generelles Vorgehen}
\label{chapter:richtlinien}
%TODO Julian
%-http://www.information-management.com/specialreports/20040518/1003611-1.html
%-http://searchsmbstorage.techtarget.com/tip/Data-migration-strategies-and-best-practices
%-https://pure.fundp.ac.be/ws/files/168599/wcre02.pdf
%http://www.dtic.upf.edu/~jbisbal/publications/datasem97.pdf
%http://dc-pubs.dbs.uni-leipzig.de/files/Rahm2000DataCleaningProblemsand.pdf
%http://csis.pace.edu/~marchese/CS775/Proj1/legacyinfosys_directions.pdf
%http://www.oracle.com/technetwork/middleware/oedq/successful-data-migration-wp-1555708.pdf




%\section{Generelles Vorgehen}
%Unabh"angig von der gew"ahlten Strategie setzt die Durchf"uhrung einer Datenmigration drei grunds"atzliche T"atigkeiten voraus \citep{henrard-2002}. Diese dienen als Ger"ust der Umsetzung einer Migration. 

Die Datenmigration stellt einen risikoreichen Aufgabenbereich dar. Aus diesem Grund sollte diese nicht ohne eine vorhergehende Planungsphase und Struktur durchgef"uhrt werden. Ein allgemeing"ultiges Verfahren mit welchem das Risiko der Datenmigration reduziert werden kann existiert jedoch nicht \citep[S.~3]{wuLawless-1997}. Dennoch existieren verschiedene Ans"atze zur Strukturierung des Migrationsprozesses. Einer dieser Ans"atze ist das von Klaus Haller, Florian Matthes und Christopher Schulz in Zusammenarbeit mit Unternehmen aus Automobil-Industrie und dem Finanzsektor entwickelte Prozessmodell f"ur Datenmigration \citep[S.~2f.]{klausMatthesSchulz-2012}. 
\lb
Dieses Prozessmodel unterteilt die Migration in vier Abschnitte: Die Initialisierung, die Migrationsentwicklung, den Testabschnitt und die Umstellung auf das neue System. F"ur jeden dieser Abschnitte werden dabei die wichtigsten T"atigkeiten dargestellt \citep[S.~5f]{klausMatthesSchulz-2012}. Je nach Unternehmen und Migrationssituation k"onnen die T"atigkeiten variieren.
\lb
W"ahrend der Initialisierung muss die aktuelle Situation erfasst werden. Hierbei gilt es herauszufinden, um was f"ur ein Legacy-System es sich handelt, wo und wie die Daten gespeichert sind, auf was f"ur ein System migriert werden soll und welche Daten "ubertragen werden sollen \citep[S.~7]{klausMatthesSchulz-2012}. Auf dieser Grundlage kann eine Entscheidung dar"uber getroffen werden, welche Strategien f"ur die Migration verwendet werden k"onnen. Diese Information sind somit essentiell, um das weitere Vorgehen planen zu k"onnen und eine initiale Aufwandseinsch"atzung erstellen zu k"onnen \citep[S.~7]{klausMatthesSchulz-2012}. Ebenfalls wird auf Grundlage der gesammelten Informationen eine Migrationsplattform eingerichtet. Mit Hilfe dieser Migrationsplattform werden im folgenden die ben"otigten Migrationswerkzeuge, -skripte usw. entwickelt und getestet \citep[S.~7]{klausMatthesSchulz-2012}.
\lb
Im Abschnitt zwei, der Migrationsentwicklung, wird zun"achst die vorangegange Analyse der auf dem Legacy-System vorhandenen Daten vertieft. Zu diesem Zweck wird ein Backup des Legacy-Datenbestands auf die Migrationsplatform "uberspielt. Somit kann verhindert werden, dass w"ahrend der Analyse Probleme oder St"orungen auf dem Legacy-System auftreten \citep[S.~7]{klausMatthesSchulz-2012}. Bei der Analyse ist festzustellen, welche Datenschemata und -formate auf dem Legacy-System vorhanden sind und auf dem angepassten oder neu entwickelten System vorhanden sein m"ussen. Besonders wichtig ist hierbei herauszufinden, worin sich diese Schemata voneinander unterscheiden. Dieses Wissen erm"oglicht es, Migrationsskripte zu entwickeln, mit welchen die Daten automatisiert konvertiert und "ubertragen werden k"onnen \citep[S.~7f.]{klausMatthesSchulz-2012}. 
\lb
Nachdem die Struktur der Daten analysiert worden ist, m"ussen die Daten selbst betrachtet werden. Man versucht auf diese Weise unvollst"andige, inkonsistente oder doppelte Datens"atze zu identifizieren. Um Probleme im neuen System zu vermeiden, sollten solche Datens"atze bereinigt werden. Die Bereinigung (engl.: \textit{ Cleansing}) kann entweder vor der Datenmigration auf dem Legacy-System durchgef"uhrt werden, w"ahrend der Migration mit Hilfe der Migrationsskripte oder im Anschluss auf dem neuen System \citep[S~7f.]{klausMatthesSchulz-2012}. Mittels dieser Bereinigung werden die inkonsisten und unvollst"andigen Datens"atze repariert und redundante Datens"atze entfernt \citep[S.~7f.]{rahm-2010}. 
\lb
Die in der Migrationsentwicklung entwickelten Skripte m"ussen anschlie"send im Testabschnitt ausgiebig auf ihre Funktionalit"at "uberpr"uft werden. Entsprechende Tests werden auch hier auf der Migrationsplattform ausgef"uhrt \citep[S.~8f.]{klausMatthesSchulz-2012}. Gefundene Fehler werden korrigiert, Migrationsskripte m"ussen "uberpr"uft werden. Dieser Prozess wird solange wiederholt, bis die Migrationsskripte fehlerfrei sind und deren Ergebnisse als korrekt validiert wurden \citep[S.~8f.]{klausMatthesSchulz-2012}.
\lb
Abschlie"send folgt die Umstellung auf das neue System. Dieser Abschnitt beinhaltet die Migration vom Legacy- auf das neue System mit Hilfe der zuvor entwickelten Migrationsskripte. Je nach gew"ahlter Vorgehensweise bleibt das Legacy-System bis zum Abschluss der Migration und Sicherstellung der Funktionalit"at des zuk"unftigen Systems im Einsatz \citep[S.~107]{bisbal-1999}. 
\lb
Nachdem die Datenmigration abgeschlossen wurde, ist es in der Regel sinnvoll, einen Erfahrungsbericht zu erstellen. In diesem soll festhalten werden, in welchen Phasen der Migration es zu Problemen kam und wie diese behoben wurden. Auf diese Weise soll der Migrationsprozess im Unternehmen aktiv verbessert werden. Ebenfalls soll sichergestellt werden, dass in zuk"unftigen Migrationsprojekten nicht dieselben Probleme auftreten \citep[S.~10]{klausMatthesSchulz-2012}.


