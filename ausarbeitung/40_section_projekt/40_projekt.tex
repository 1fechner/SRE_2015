% !TeX spellcheck = de_DE
\section{Allgemeine Planungsrichtlinien}
\label{chapter:richtlinien}
%TODO Julian
%-http://www.information-management.com/specialreports/20040518/1003611-1.html
%-http://searchsmbstorage.techtarget.com/tip/Data-migration-strategies-and-best-practices
%-https://pure.fundp.ac.be/ws/files/168599/wcre02.pdf
%http://www.dtic.upf.edu/~jbisbal/publications/datasem97.pdf
%http://dc-pubs.dbs.uni-leipzig.de/files/Rahm2000DataCleaningProblemsand.pdf
%http://csis.pace.edu/~marchese/CS775/Proj1/legacyinfosys_directions.pdf
%http://www.oracle.com/technetwork/middleware/oedq/successful-data-migration-wp-1555708.pdf

Die Datenmigration stellt einen risikoreichen Aufgabenbereich dar. Aus diesem Grund sollte diese nicht ohne eine vorhergehende Planungsphase durchgeführt werden. Es existiert keine Standardvorgehensweise für diese Planungsphase \cite[S.~3]{wuLawless-1997}. Stattdessen existieren es verschiedene Sammlungen von allgemeinen Richtlinien zur erfolgreichen Durchführung der Datenmigration. In Folgenden werden einige dieser Richtlinie angef"uhrt.

\subsection{MILESTONE-Richtlinien}

% Hier Grafik Übersicht MILESTONE

Das MILESTONE-Projekt stellte eine Kollaboration des Trinity College Dublin, Broadcom Éireann Research, Telecom Éireann und Ericssons dar. Es begann 1996 und endete 1998. Im Zuge dieses Projektes sollten Migrations-Methoden und diese Methoden unterstützende Werkzeuge entwickelt werden \cite[S.~3]{wuLawless-1997}. Im Zuge des MILESTONE-Projektes wurde, unter anderen, die Butterfly-Vorgehensweise zur Datenmigration entwickelt, welche in Kapitel \ref{chapter:vorgehensweisen} besprochen wird \cite[S.~3]{wuLawless-1997}.
\lb
Im MILESTONE-Projekt unterteilt man den Migrationsprozess in fünf Aufgabenbereiche: Rechtfertigung (engl.: \textit{Justification}), Verstehen des Legacy-Systems (engl.: \textit{Legacy System Understanding}), Entwicklung des Zielsystems (engl.: \textit{Target System Development}), Migration und Testen \cite[S.~3]{wuLawless-1997}. Jeder dieser Aufgabenbereiche stellt ein notwendiges Erfordernis für den Erfolg der Migration dar. So wird innerhalb des Aufgabenbereiches der Rechtfertigung die evaluiert, ob die Entwicklung eines neuen Systems sinnvoll ist oder ob das urspr"ungliche System weiter genutzt werden kann. Um diese Fragestellung zu klären, müssen zunächst die Vorteile und auch die Risiken des Legacy-Systems sowie des potentiell neu zu entwickelnden Systems gesammelt und einander gegenüber gestellt werden \cite[S.~3]{wuLawless-1997}.
\lb
Ist geklärt, ob der Einsatz eines neuen Systems oder eine Anpassung gerechtfertigt ist, kann mit den weiteren Aufgabenbereichen begonnen werden. Zum verstehen des Legacy-Systems muss dieses und dessen zugehörige Dokumentation, sofern vorhanden, analysiert werden \cite[S.~3f.]{wuLawless-1997}. Hierbei ist muss geklärt werden was für Daten auf den Legacy-System und wie diese gespeichert sind. Dafür kommen Techniken des Reverse Engineering zum Einsatz \cite[S.~3f.]{wuLawless-1997}. In dieser Phase müssen alle Funktionen und die vom System zu deren Ausführung benötigten Daten identifiziert werden. Des Weiteren muss der Datenbestand des Legacy-Systems auf redundante, unvollständige und veraltete Datensätze untersucht werden \cite[S.~3f.]{wuLawless-1997}.
\lb
Sobald alle Daten und Funktionalitäten des Legacy-Systems identifiziert wurden, kann mit dem nächsten Aufgabenbereich, der Entwicklung des Zielsystems oder der Anpassung, begonnen werden. Ziel ist es, ein  System zu entwickeln, welches alle Funktionalit"aten des Legacy-Systems abdeckt. Darüber hinaus m"ussen im neuen System Nachteile des Legacy-Systems, etwa schlechte Wartbarkeit, beseitigt werden \cite[S.~3ff.]{wuLawless-1997}.
\lb
Im nächsten Schritts der Migration nach MILESTONE findet die Migration statt. Abhängig von der gewählten Strategie verläuft diese Phase unterschiedlich. Prinzipiell soll der Umzug vom Legacy- auf das neue System durchgeführt werden. Dies beinhaltet auch die Migration der Daten und der Anwendungen, ebenso das Einarbeiten der Benutzer auf dem neuen System \cite[S.~3ff.]{wuLawless-1997}.
\lb
Anderes als die vorherigen Aufgabenbereiche, die lediglich durch ihre Ergebnisse miteinander interagieren, überwacht das Testen sämtliche Aktivitäten der einzelnen Phasen jenseits der Rechtfertigung. Durch die rechtzeitige Durchführung von Tests in jeder Phase sollen mögliche Fehler und Ungenauigkeiten identifiziert und behoben werden \cite[S. 3f.]{wuLawless-1997}.

%TODO passendes Ende noch ergänzen

\subsection{4-Phasen-Modell von Haller, Matthes und Schulz}

%TODO Hier noch Grafik einfügen
%TODO Stages statt Phasen!!!

Das 4-Phasen-Modell von Haller, Matthes und Schulz wurde in Zusammenarbeit mit mehreren Unternehmen aus der Automobil-Industrie, dem Finanzsektor sowie weiteren aus anderen Sektoren entwickelt \citep[S.~2f.]{klausMatthesSchulz-2012}. Es soll der Modellierung des Prozesses der Datenmigration dienen. Hierbei liegt das Hauptaugenmerk der Autoren auf der Einteilung des Migrationsprozesses in vier Phasen sowie deren einzelne Aufgaben, Zuständigkeiten und Ergebnissen \citep[S.~5f.]{klausMatthesSchulz-2012}.
\lb
Die erste Phase von den Autoren als Initialisierung bezeichnet. In dieser Phase wird zunächst geklärt, wer die bevorstehende Datenmigration durchführen soll. Zur Auswahl stehen hier etwa die hauseigene IT-Abteilung oder ein externer IT-Dienstleister \citep[S.~7]{klausMatthesSchulz-2012}. Nachdem eine Entscheidung "uber die Zust"andigkeit getroffen wurde, kann mit der ersten Analyse des bestehenden Legacy-Systems begonnen werden. Ziel ist auf Basis dieser ersten Analyse eine Erkenntnis dar"uber, welche Daten vorhanden sind, welche der bestehenden Daten migriert werden sollen und welche verworfen werden können \citep[S.~7]{klausMatthesSchulz-2012}. Des Weiteren müssen vorhandene organisatorische und technische Einschränkungen identifiziert werden, wie z.B eine bereits bestehende Datenbank welche ins neue System integriert werden soll oder das Vorhandensein von an mehreren Standorten verteilten Teams. Auch muss festgelegt werden, welche Abnahmekriterien das neue System erfüllen muss \citep[S.~7]{klausMatthesSchulz-2012}. Auf dieser Grundlage muss anschließend auch die eingesetzte Strategie der Datenmigration festgelegt werden \citep{klausMatthesSchulz-2012}. Ein dritter Aufgabenbereich, welcher  w"ahrend der Phase der Initialisierung abgearbeitet werden muss, ist das Aufsetzen der Datenmigrationsplattform. Diese Datenmigrationsplattform dient im Folgenden dem Datenmigrationsteam der Entwicklung und dem Testen der benötigten Werkzeuge, wie z.B. den Migration-Skripten. Darüber hinaus bildet die Datenmigrationsplattform ebenfalls eine Grundlage für die Testmigrationen \citep[S.~7]{klausMatthesSchulz-2012}.
\lb
In der zweiten Phase, der Migrationsentwicklung (engl.: \textit{Migration Development}) wird zunächst eine Kopie des aktuellen Datenbestands des Legacy-Systems auf der Migrationsplattform angelegt. Mittels dieser Kopie können die Migrationsskripte auf der Migrationsplattform entwickelt und getestet werden, ohne das Einschränkungen oder Probleme für die Ablaufe im Unternehmen entstehen \citep[S.~7]{klausMatthesSchulz-2012}. Im Anschluss daran müssen die Daten und die Struktur des Legacy-Systems eingehender untersucht werden. Während sich in Phase eins nur ein grober Überblick darüber verschafft wurde, welche Daten im Legacy-System vorhanden sind, muss jetzt die Struktur der Daten und deren Qualität betrachtet werden \citep[S.~7f.]{klausMatthesSchulz-2012}. Bei der Analyse der Struktur muss überprüft werden, wie Daten persistiert wurden. Es wird erfasst, welche Tabellen in einer Datenbank existieren und über welche Attribute diese verfügen. Die hier gewonnenen Erkenntnisse müssen anschlie"send mit dem neuen System abgeglichen werden, damit während der Entwicklung auf die Abweichungen entsprechend eingegangen und somit sp"atere Fehler vermieden werden können \citep[S.~8]{klausMatthesSchulz-2012}. Durch die Analyse der Datenqualit"at werden unvollst"andige oder fehlende Daten, welche zunächst repariert beziehungsweise ergänzt werden m"ussen, identifiziert. Dies kann beispielsweise durch Erg"anzen zusätzlicher Informationen oder durch L"oschen der besch"adigten Daten w"ahrend oder vor der Migration erfolgen \citep[S.~7f.]{klausMatthesSchulz-2012}. Als letztes findet in dieser Phase die Daten=Transformation (engl.: \textit{Data Transformation}) statt. Ziel ist es hierbei, die Transformationsregeln für die Migrationsskripte aus den gewonnenen Erkenntnissen der Struktur- und Qualitätsanalyse zu generieren. In diesen Transformationsregeln wird festgelegt, wie Inhalte der urspr"unglichen Datenquelle bzw. deren Attribute im neuen System hinterlegt werden sollen \citep[S.~8]{klausMatthesSchulz-2012}.
\lb
Nachdem die Migrationsskripte in Phase zwei entwickelt worden sind, k"onnen diese nun in Phase drei, der Testphase (eng.: \textit{Testing}) ausführlich getestet werden. Neben einen Migrationstestlauf auf der Migrationsplattform werden noch zus"tzlich ei e Erscheinungstest (engl.: \textit{Appearance Test}), ein Vollständigkeits- und Typen-"Ubereinstimmungstest (engl.: \textit{Completeness \& Type Correspondent Test}), eine Verarbeitungstest (engl.: \textit{Processability Test}) sowie ein Integrationstest mit denen die Testmigration validiert werden soll \citep[S.~8f.]{klausMatthesSchulz-2012}. So wird mittels des Erscheinungstests überprüft, ob die Daten auch korrekt "ubertragen werden \citep[S.~8f.]{klausMatthesSchulz-2012}. Der Vollständigkeits- und Typ-Übereinstimmungstest soll herausfinden, ob Daten während der Migration verloren gegangen sind oder sich nicht mehr denn richtigen Objekttyp, wie z.B. Kunde, zugeordnet werden k"onnen \citep[S.~9]{klausMatthesSchulz-2012}. Verarbeitungs- und Integrationstest "uberpr"ufen, ob die migrierten Daten auch korrekt vom neuem System verarbeitet werden können und ob das neue System auch mit den anderen Systemen des Unternehmen interagieren kann \citep[S.~9]{klausMatthesSchulz-2012}. Bei Auftreten von Fehlern muss deren Ursprung festgestellt werden und in den Migrationsskripten behoben werden. Die korrigierten Migrationsskripte müssen anschließend erneut getestet werden. Erst wenn alle Tests erfolgreich durchlaufen wurden, kann die Generalprobe (engl.: \textit{Final Rehearsal}) der Datenmigration beginnen. Hierbei wird eine Testmigration durchgeführt, in diesem Fall jedoch unter realistischen Bedingungen Treten auch hier keine weiteren Auffälligkeiten mehr auf, gilt die Testphase als erfolgreich abgeschlossen \citep[S.~9f.]{klausMatthesSchulz-2012}.
\lb
Die letzte Phase des 4-Phasen-Modells von Haller, Matthes und Schulz stellt den Übergang von Legacy- hin zum neuen System dar. In dieser Phase wird nicht mehr mit der Migrationsplattform gearbeitet, stattdessen werden die Migrationsskripte direkt auf das Legacy-System angewandt um die Daten auf das neue System zu migrieren \citep[S.~10]{klausMatthesSchulz-2012}. Je nach gewählter Strategie kann das Legacy-System bis zum Abschluss der Migration noch aktiv im Unternehmen genutzt werden. 
%TODO Belegendes Zitat und vllt. Ausführung?
Nachdem die Datenmigration abgeschlossen wurde, ist es in der Regel sinnvoll, noch ein Erfahrungsbericht zu erstellen. In diesem Erfahrungsbericht soll festhalten werden, in welchen Phasen der Migration es zu Probleme kam und wie diese behoben wurden. Auf diese Weise soll der Migrationsprozess im Unternehmen aktiv verbessert werden. Ebenfalls soll sichergestellt werden, dass in zuk"unftigen Migrationsprojekten nicht dieselben Probleme auftreten \citep[S.~10]{klausMatthesSchulz-2012}.

%TODO passendes Ende noch ergänzen