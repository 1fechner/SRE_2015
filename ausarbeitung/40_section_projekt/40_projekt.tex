% !TeX spellcheck = de_DE
\section{Planung und Durchführung}
%TODO Julian
%http://www.information-management.com/specialreports/20040518/1003611-1.html
%http://searchsmbstorage.techtarget.com/tip/Data-migration-strategies-and-best-practices
%https://pure.fundp.ac.be/ws/files/168599/wcre02.pdf
%http://www.dtic.upf.edu/~jbisbal/publications/datasem97.pdf
%http://dc-pubs.dbs.uni-leipzig.de/files/Rahm2000DataCleaningProblemsand.pdf
%http://csis.pace.edu/~marchese/CS775/Proj1/legacyinfosys_directions.pdf
%http://www.oracle.com/technetwork/middleware/oedq/successful-data-migration-wp-1555708.pdf

\begin{itemize}
	\item Vorstellung der einzelnen Phasen der Datenmigration
	\begin{itemize}
		\item Data Assessment
%		\begin{itemize}
%			\item Welche Datenquellen existieren
%			\item Welche Daten müssen migriert werden
%			\item Wie wird migriert(Methode) und wie wird anschließend die Validation durchgeführt
%		\end{itemize}
		
		\item Data Cleansing
%		\begin{itemize}
%			\item Was wird nicht ben"otigt
%			\item Aufr"aumen der bestehenden Daten
%			\item "Uberpr"ufen der Datenqualität
%		\end{itemize}
		
		\item Test Extract \& Load
%		\begin{itemize}
%			\item Erstellen der notwendigen Skripte zur automatischen Durchf"uhrung der Datenmigeration 
%			\item Testlauf mit Teildatensatz
%		\end{itemize}
		
		\item Final Extract \& Load
%		\begin{itemize}
%			\item Durchf"uhrung der Datenmigration mit jeweiliger Vorgehensweise
%		\end{itemize}
		
		\item Migration Validation
%		\begin{itemize}
%			\item Abschließende "Uberpr"ufung der migrierten Daten
%			\item eventuelle Korrekturen
%		\end{itemize}
		
		\item Post Migration Activities
%		\begin{itemize}
%			\item Anpassung Dokumentation und Referenzen
%			\item Lernen fürs nächste Mal
%		\end{itemize}
		
	\end{itemize}
\end{itemize}