% !TeX spellcheck = de_DE
\section{Allgemeine Planungsrichtlinien}
%TODO Julian
%-http://www.information-management.com/specialreports/20040518/1003611-1.html
%-http://searchsmbstorage.techtarget.com/tip/Data-migration-strategies-and-best-practices
%-https://pure.fundp.ac.be/ws/files/168599/wcre02.pdf
%http://www.dtic.upf.edu/~jbisbal/publications/datasem97.pdf
%http://dc-pubs.dbs.uni-leipzig.de/files/Rahm2000DataCleaningProblemsand.pdf
%http://csis.pace.edu/~marchese/CS775/Proj1/legacyinfosys_directions.pdf
%http://www.oracle.com/technetwork/middleware/oedq/successful-data-migration-wp-1555708.pdf

Die Datenmigration stellt einen risikoreichen Aufgabenbereich dar. Aus diesem Grund sollte diese auch nicht ohne eine vorhergehende Planungsphase durchgeführt werden. Allerdings keine Standardvorgehensweise für diese Planungsphase.\cite[S. 3]{wuLawless-1997} Stattdessen gibt es verschiedene Sammlungen von allgemeinen Richtlinien zur erfolgreichen Durchführung der Datenmigration. In folgendem werden ein paar dieser Richtlinie besprochen.

\subsection{MILESTONE-Richtlinien}

% Hier Grafik Übersicht MILESTONE

Das MILESTONE-Projekt stellte eine Kollaboration des Trinity College Dublin, Broadcom Éireann Research, Telecom Éireann und Ericssons dar. Es begann 1996 und endete 1998. Im Zuge dieses Projektes sollten Migrations-Methoden und diese Methoden unterstützende Werkzeuge entwickelt werden.\cite[S. 3]{wuLawless-1997} Unter anderen wurde im MILESTONE-Projekt auch die Butterfly-Vorgehensweise zur Datenmigration entwickelt, die in Kapitel 5 besprochen wird.\cite[S. 3]{wuLawless-1997} 
\lb
Im MILESTONE-Projekt unterteilt man den Migrationsprozess in fünf Aufgabenbereiche: Rechtfertigung(engl.: Justification), Verstehen des Legacy-Systems(engl.: Legacy System Understanding), Entwicklung des Zielsystems(engl.: Target System Development), Migration und Testen.\cite[S. 3]{wuLawless-1997} Jede dieser Aufgabenbereiche stellt eine notwendiges Erfordernis für den Erfolg der Migration dar. So wird innerhalb des Aufgabenbereiches der Rechtfertigung die Fragestellung geklärt, ob die Entwicklung eines neuen Systems sinnvoll ist oder ob noch weiter mit dem Alten weitergearbeitet werden soll. Um diese Fragestellung zu klären, müssen zunächst die Vorteile und auch die Risiken des Legacy System sowie des potentiell neu zu entwickelnden System gesammelt und einander gegenüber gestellt werden.\cite[S. 3]{wuLawless-1997}
\lb
Wenn geklärt ist ob der Einsatz eines neuen Systems gerechtfertigt ist, kann mit den weiteren Aufgabenbereichen begonnen werden. Zum verstehen des Legacy-Systems muss dieses und dessen zugehörige Dokumentation, sofern vorhanden, analysiert werden.\cite[S. 3f]{wuLawless-1997} Hierbei ist muss geklärt werden was für Daten auf den Legacy-System und wie diese gespeichert sind. Dafür kommen Techniken des Reverse Engineering zum Einsatz.\cite[S. 3f]{wuLawless-1997} In dieser Phase müssen alle Funktionen und die vom System zu deren Ausführung benötigten Daten identifiziert werden. Des Weiteren muss der Datenbestand des Legacy-System auf redundante, unvollständige und veraltete Datensätze untersucht werden.\cite[S. 3f]{wuLawless-1997}
\lb
Sobald nun alle Daten und Funktionalitäten des Legacy-Systems identifiziert wurden, kann mit der nächsten Aufgabenbereich, der Entwicklung des Zielstystems, begonnen werden. Ziel ist es hier ein neues System zu entwickeln, welches alle Funktionen des Legacy-Systems abdeckt. Darüber hinaus sollen im neuen System jedoch Nachteile des Legacy-Systems, wie z.B. schlechte Wartbarkeit, beseitigt werden.\cite[S. 3ff]{wuLawless-1997} 
\lb
Im nächsten Aufgabenbereich der Migration nach MILESTONE findet die Migration statt. Abhängig von der gewählten Vorgehensweise verläuft dieser Aufgabenbereich unterschiedlich. Prinzipiell soll hier aber der Umzug vom Legacy- auf das neue System durchgeführt werden. Das beinhaltet auch die Migration der Daten und der Anwendungen, aber auch das Einarbeiten der Benutuer auf dem neuen System.\cite[S. 3ff]{wuLawless-1997}
\lb
Anderes als die vorherigen Aufgabenbereiche, die sich nur durch ihre Ergebnisse miteinander interagieren, überwacht das Testen sämtliche Aktivitäten der einzelnen Phasen jenseits der Rechtfertigung. Durch die rechtzeitige Durchführung Tests in jeder Aufgabenbereiche sollen mögliche Fehler und Ungenauigkeit identifiziert  und behoben werden.\cite[S. 3f]{wuLawless-1997}

% passendes Ende noch ergänzen

\subsection{4-Phasen-Modell von Haller, Matthes und Schulz}