% !TeX spellcheck = de_DE
\section{Fazit}
\label{chapter:fazit}

Die Datenmigration im Kontext des Reengineering ist kein triviales Verfahren. Durch die Vielschichtigkeit von Unternehmen und Softwaresystemen eignen sich in unterschiedlichen Kontexten verschiedene Strategien. Als h"aufig untersch"atztes Verfahren hat die Migration von Daten einen hohen Gesch"aftswert, etwa die Erhaltung gesch"aftskritischer Kundendaten. Mangelnde Planung und undefinierte Herangehensweisen k"onnen Projekte der Datenmigration scheitern lassen. Durch das Nutzen von Strategien zur Durchf"uhrung, Planungsrichtlinien und Einf"uhrungsstrategien kann die Migration strukturiert werden.
\lb
Sowhol die Migration der Daten selbst, als auch die Anpassung umliegender Anwendungen und Softwaresysteme spielt eine zentrale Rolle. Unterschiedliche Konzepte bieten jeweils fundierte Strategien f"ur die Migration. Aspekte wie fachliches und technisches Wissen um Schnittstellen und Datenmodelle k"onnen Grundlage f"ur die Auswahl einer Strategie sein. Der zuk"unftige Gesch"aftswert nach durchgef"uhrten "Anderungen kann bei der Auswahl ebenfalls eine Rolle spielen. So k"onnen neue technische M"oglichkeiten nach Austausch der Datenbank-Plattform effizientere Schnittstellen bereitstellen und Zugriffe redundanzfrei gestalten.
\lb
%TODO Julian dein Part... Lass uns das auf eine Seite beschr"anken
\lb
Die Durchf"uhrung der Datenmigration ist ein dynamisches Verfahren. Im speziellen Kontext von Unternehmen, Projektteam und vorhandenen Systemen sind zu nutzenden Strategien individuell abzustimmen. Gerade die Ungewissheit, welche Legacy-System h"aufig mit sich bringen, erm"oglichen keine statische Auswahl einer Musterl"osung f"ur die Datenmigration.