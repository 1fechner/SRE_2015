% !TeX spellcheck = de_DE
\section{Fazit}
\label{chapter:fazit}

Die Datenmigration im Kontext des Reengineering ist kein triviales Verfahren. Durch die Vielschichtigkeit von Unternehmen und Softwaresystemen eignen sich in unterschiedlichen Kontexten verschiedene Strategien. Als h"aufig untersch"atztes Verfahren hat die Migration von Daten einen hohen Gesch"aftswert, etwa die Erhaltung gesch"aftskritischer Kundendaten. Mangelnde Planung und undefinierte Herangehensweisen k"onnen Projekte der Datenmigration scheitern lassen. Durch das Nutzen von Strategien zur Durchf"uhrung, Planungsrichtlinien und Einf"uhrungsstrategien kann die Migration strukturiert werden.
\lb
Sowohl die Migration der Daten selbst, als auch die Anpassung umliegender Anwendungen und Softwaresysteme spielt eine zentrale Rolle. Unterschiedliche Konzepte bieten jeweils fundierte Strategien f"ur die Migration. Aspekte wie fachliches und technisches Wissen um Schnittstellen und Datenmodelle k"onnen Grundlage f"ur die Auswahl einer Strategie sein. Der zuk"unftige Gesch"aftswert nach durchgef"uhrten "Anderungen kann bei der Auswahl ebenfalls eine Rolle spielen. So k"onnen neue technische M"oglichkeiten nach Austausch der Datenbank-Plattform effizientere Schnittstellen bereitstellen und Zugriffe redundanzfrei gestalten.
\lb
Bei der Durchf"uhrung und bei der Wahl einer Strategie ist stets zu beachten, dass es keine universelle L"osung gibt, welche immer das beste Ergebnis liefert. Vielmehr muss f"ur jedes Unternehmen jede m"ogliche Vorgehensweise genau betrachtet werden. So bietet sich beispielsweise der Chicken-Little-Ansatz f"ur ein Unternehmen an, welches eine Legacy-Anwendungen St"uck f"ur St"uck nacheinander ersetzen will, um das allt"agliche Gesch"aft nicht zu gef"ahrden. Wenn das Unternehmen allerdings plant, ein System komplett zu ersetzen und dabei Ausfallzeiten minimal halten will, eignet sich wom"oglich der Butterfly-Ansatz besser. 
\lb
Um die richtige Entscheidung treffen zu k"onnen, muss immer die Gesamtsituation analysiert werden. Was f"ur ein Unternehmen liegt vor, sollen nur die Daten auf ein neues System migriert werden oder muss das neue System auch erst noch entwickelt werden? F"ur diese Fragen und weitere m"ussen Antworten gefunden werden. Aus diesem Grund stellt ein geplantes Vorgehen eine wichtige Rolle f"ur die erfolgreiche Datenmigration. Aber auch hierbei ist anzumerken, dass keine allgemeing"ultige Planung existiert und diese an diese Gegebenheiten angepasst werden muss. 
\lb
Die Durchf"uhrung der Datenmigration ist ein dynamisches Verfahren. Im speziellen Kontext von Unternehmen, Projektteam und vorhandenen Systemen sind zu nutzenden Strategien individuell abzustimmen. Gerade die Ungewissheit, welche Legacy-System h"aufig mit sich bringen, erm"oglichen keine statische Auswahl einer Musterl"osung f"ur die Datenmigration.